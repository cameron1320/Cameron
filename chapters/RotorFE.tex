\chapter{Finite Element Method For Rotordynamic systems}
\section{Beam Element FE Equation}
\subsection{Bernoulli-Euler Beam equation}

Assumptions used to derive the Bernoulli-Euler beam equation are (Complete derivation in \cite{craig2006fundamentals}):
\begin{enumerate}
	\item Beam is bending in a plane, in this case in the y-direction, where the x-direction is along the length of the beam.
	\item The  neutral axis undergoes no deformation in the longitudinal direction.
	\item Cross sections remain plane and perpendicular to the nuetral axis.
	\item The material is linear-elastic.
	\item Stresses in the y and z direction are negligible compared to those in the x direction.
	\item Rotary inertial effects are not considered.
	\item Mass density is constant at each cross section, so that each mass center is coincident with the centroid of that section.
\end{enumerate}
Using kinematics and assumptions 2 \& 3, the strain in the x direction may be related to the curvature of the beam, $\mu(x,t)$, and the distance from the neutral axis by
\begin{equation} \label{eq:curvature}
\epsilon=-\dfrac{y}{\mu}
\end{equation}
then, with assumption 4 \& 7 the relation from curvature to moment is
\begin{equation} \label{eq:curv_moment}
M(x,t)=\dfrac{EI}{\mu}
\end{equation}
\begin{figure}
	\centering
	\def\svgwidth{300pt}
	\import{figures/}{BeamFBD.pdf_tex}
	\caption{Free body diagram of a beam section in planar bending.}
	\label{fig:BeamFBD}
\end{figure}
where $ E $, Young's modulus, and $ I $, area moment of inertia are constant in cross sections. By using Newton's laws and the free body diagram of a single beam element, see Figure \ref{fig:BeamFBD}, the equations of motion are summarized as:
\begin{equation} \label{eq:Newton}
\sum{F_y}=\Delta m \ddot{v} \quad \textrm{\&} \quad \sum{M_G}=0
\end{equation}
Moment equation is represented as moments summarized at the center of mass, $ G $. The right hand side of moment equation of EOM \eqref{eq:Newton} is know to be null due to assumption 6. Applying Newton's equations \eqref{eq:Newton} to the FBD of Figure \ref{fig:BeamFBD} results in the force equation
\begin{equation} \label{eq:force_EOM}
F(x,t)-F(x+\Delta x,t)=\rho A\Delta x \frac{\partial^2 v}{\partial t^2}
\end{equation}
and moment equation
\begin{equation} \label{eq:moment_EOM}
-M(x,t) + M(x+\Delta x,t) + F(x,t)\left( \frac{-\Delta x}{2}\right)  + \left[-F(x+\Delta x,t)\right] \left(\frac{\Delta x}{2}\right) = 0
\end{equation}
Taking the limit of equations \eqref{eq:force_EOM} \& \eqref{eq:moment_EOM} as $ \Delta x \rightarrow 0 $ results in equations \eqref{eq:force_EOM_partial} \& \eqref{eq:moment_EOM_partial} respectively.
\begin{equation} \label{eq:force_EOM_partial}
\frac{\partial F}{\partial x}=-\rho A \frac{\partial^2 v}{\partial t^2}
\end{equation} 
\begin{equation} \label{eq:moment_EOM_partial}
\frac{\partial M}{\partial x} - F=0
\end{equation}
Assuming the beam slope, $ \frac{\partial v}{\partial x} $, remains relatively small, then linearized curvature of the beam is inversely related to $ \frac{\partial^2 v}{\partial x^2} $. Substituting this linearized curvature in \eqref{eq:curv_moment} produces
\begin{equation} \label{eq:moment_curvature_partial}
M(x,t)=EI\frac{\partial^2 v}{\partial x^2}
\end{equation}
Using linearized moment equation \eqref{eq:moment_curvature_partial}, combined with \eqref{eq:force_EOM_partial} \&  \eqref{eq:moment_EOM_partial} lends the Euler beam equation
\begin{equation} \label{eq:euler_beam_equation}
\frac{\partial^2}{\partial x^2} \left(EI\frac{\partial^2 v}{\partial x^2}\right) = -\rho A\frac{\partial^2 v}{\partial t^2}
\end{equation}
This is the governing differential equation for transverse motion of a slender beam. This equation is not suitable for an application involving lengths that are not much greater than the width of the beam \cite{genta2007dynamics}.
\sectionbreak
\subsection{Timoshenko Beam Finite Element}
The Timoshenko beam element allows for the beam cross section plane at any axis location to differ from normal with the axis of the beam. In other words, the element allows for shear stresses. This element often also includes the effects of rotary inertia and gyroscopic moments, as it will in this derivation. The generalized displacements used are assumed to be variable in both time and space.
\subsubsection{Kinematic Relationships}


\begin{figure}
	\centering
	\def\svgwidth{400pt}
	\import{figures/}{kinematicbeamsurface.pdf_tex}
	\caption{Beam Element with nodal displacements.}
	\label{fig:KineBeamElem}
\end{figure}
\begin{figure}
	\centering
	\def\svgwidth{250pt}
	\import{figures/}{TimoBeamDOF.pdf_tex}
	\caption{Timoshenko beam section with degrees of freedom at some point x along beam axis.}
	\label{fig:TimoBeamDOF}
\end{figure}
Motion of two points is taken into consideration on at some arbitrary point on the beam axis x, show in Figure \ref{fig:KineBeamElem}. The first point is $ C $ which falls on the center of the beam cross section at x, and in the undeformed configuration, the vector $ \vec{r}_c^\prime= \overrightarrow{OC} $ forms a right angle with the surface of the cross section. The second point, $ P $ is at some arbitrary $ (y,z) $ location on the cross section. The vecor $ \vec{t}=\overrightarrow{CP} $ points from the beam axis to the point along the cross section. If we follow this point $ P $ we will be able to define the motion of the cross section as a whole. The goal is to define the displacements $ u,v,w $. The motion of point P is split into translation and rotation, where $ \vec{t} $ is rotated and $ \vec{r}_c $ is translated to point to the deformed location $ P' $. The vector pointing to P in the undeformed state is defined as
\begin{equation}\label{eq:r}
\vec{r}=\vec{r}_c+\vec{t}
\end{equation}
Rotations are represented with a rotation transformation matrix,
\begin{equation}\label{eq:trot}
\vec{t}^\prime=\underline{R}\vec{t}
\end{equation}
The linearized first order rotational matrix for small angles is represented by 
\begin{equation}\label{eq:RotTransformation}
\underline{R}=\left[\begin{array}{ccc}
1&-\theta&\psi\\
\theta&1&-\phi\\
-\psi&\phi&1
\end{array}\right]
\end{equation}
and the translation with a displacement vector 
\begin{equation}\label{eq:rtrans}
\vec{r}_c^\prime=\vec{r}_c+\vec{u}_c
\end{equation}
%\begin{equation}\label{eq:RotTransformationExpanded}
%R=\underline{I}+\tilde{\underline{\Psi}}+1/2\left(\tilde{\underline{\Psi}}\right)^2+...
%\end{equation}
%\begin{equation}\label{eq:RotTransformationApprox}
%R\approx\underline{I}+\tilde{\underline{\Psi}}
%\end{equation}
Combined motion from $ P $ to $ P' $ can be defined by
\begin{equation}\label{eq:DispVectorrpr}
\vec{u}=\vec{r^\prime}-\vec{r}
\end{equation}
where the vector definitions for $ \vec{r}' $ and $ \vec{r} $ are substituted to the above equation to obtain
\begin{equation}\label{eq:DispVectorrptprt}
\vec{u}=\vec{r_c^\prime}+\vec{t^\prime}-\vec{r}_c+\vec{t}
\end{equation}
and now using the definition for $ \vec{u}_c $ and $ \vec{t}' $ leads to the simplified expression for the motion of $ P $
\begin{equation}\label{eq:DispVectRotTranst}
\vec{u}=\vec{u}_c+(\underline{R}-\underline{I})\vec{t}
\end{equation}

expanding the matrices reveals the system as
\begin{equation}\label{DispVectExpanded}
\vec{u} = \left\{\begin{array}{c}
	u\\
	v\\
	w\end{array}\right\}=\left\{\begin{array}{c}
	u_c\\
	v_c\\
	w_c\end{array}\right\}+\left[\begin{array}{ccc}
	0&-\theta&\psi\\
	\theta&0&-\phi\\
	-\psi&\phi&0
	\end{array}\right]\left\{\begin{array}{c}
	0\\
	y\\
	z\end{array}\right\}
\end{equation}
Therefore, the motion of any point on the beam may be approximated with
\begin{equation}\label{eq:DispVectEvaluated}
\vec{u}=\left\{\begin{array}{c}
u_c-\theta y+\psi z\\
v_c-\phi z\\
w_c+\phi y\end{array}\right\}
\end{equation}
\subsubsection{Internal Constitutive Relationship}
Stresses are assumed to exist in the beam in the axial direction, and in shear on the face of the beam section. Stresses in the transverse, or the y and z, directions are assumed negligible. Shear stresses out of the plane section are assumed to be vanishing as the differential element shrinks. Internal damping is to be considered independently from this material constitutive relationship Written out, these stresses are represented by the matrix
\begin{equation}\label{key}
\sigma_{ij}=\left[\begin{array}{ccc}
\sigma_{xx}&\sigma_{xy}&\sigma_{xz}\\
\sigma_{xy}&0&0\\
\sigma_{xz}&0&0
\end{array}\right]
\end{equation}
using the Hooke's Law for a linear elastic isotropic material, expressed as
\begin{equation}\label{key}
\epsilon_{ij}=\frac{1}{E}[(1+\nu)\sigma_{ij}-\nu\delta_{ij}\sigma_{kk}]
\end{equation}
allows the determination of the stress strain relationship in engineering notation as
\begin{equation}\label{eq:ConstitutiveRelationshipEngineeringNotation}
\left\{\begin{array}{c}
\sigma_{xx}\\ \sigma_{xy}\\ \sigma_{xz}
\end{array}\right\} = \left[\begin{array}{ccc}
E&0&0\\0&2G&0\\0&0&2G
\end{array}\right] \left\{\begin{array}{c}
\epsilon_{xx}\\ \epsilon_{xy}\\ \epsilon_{xz}
\end{array}\right\}
\end{equation}
where, $ G=\frac{E}{2(1+\nu)} $. Strains are derived from displacements of equation \eqref{eq:DispVectEvaluated} using a linear strain-displacement relationship for infinitesimal strains: $ 2\epsilon_{ij}=u_{i,j}+u_{j,i} $.
%\begin{equation}\label{eq:LinearStrainDispRelationship}
%
%\end{equation}
%\begin{equation}\label{eq:StrainEvaluated}
%\underline{\epsilon}=\frac{1}{2}\left[\begin{array}{ccc}
%2\left(\frac{\partial u_c}{\partial x}-\frac{\partial\theta}{\partial x}y+\frac{\partial\psi}{\partial x}z\right) & \frac{\partial v_c}{\partial x}-\frac{\partial\phi}{\partial x}z-\theta & \frac{\partial w_c}{\partial x}+\frac{\partial \phi}{\partial x}y+\psi\\
%\frac{\partial v_c}{\partial x}-\frac{\partial\phi}{\partial x}z-\theta &0&0\\
%\frac{\partial w_c}{\partial x}+\frac{\partial \phi}{\partial x}y+\psi&0&0\end{array}\right]
%\end{equation}
The subscript c is dropped and $ u $,$ v $, and $ w $ are taken to be representative of the displacements on the beam axis. Since all strains will be collapsed into strain energy of the entire beam cross section. 
\begin{equation}\label{eq:StrainEvaluatedSimple}
\left\{\arraycolsep=1pt\begin{array}{rl}
\epsilon_{xx}&=u'-\theta'y+\psi'z\\
\epsilon_{xy}&=\frac{1}{2}(v'-\phi'z-\theta)\\
\epsilon_{xz}&=\frac{1}{2}(w'+\phi'y+\psi)
\end{array}\right.
\end{equation}
Note that in the case of the Euler-Bernoulli beam derivation the slope in the y direction displacement is equal to the face angle about the z-axis, i.e., $v'=\theta$ and $w'=\psi$. Application of those would reduce the system to the Euler-Bernoulli beam.\par

It will be proven useful to introduce generalized strains that group strain contributions above as axial, bending, torsion, and shear, represented by the symbols $ \varepsilon $, $ \rho $, $ \varphi $, $ \gamma $,respectively.
\begin{equation}\label{eq:GeneralizedStrains}
\left\{\arraycolsep=1pt\begin{array}{rl}
\varepsilon&=u'\\
\rho_{y}&=-\theta'\\
\rho_{z}&=\psi'\\
\varphi&=\phi'\\
\gamma_{y}&=v'-\theta\\
\gamma_{z}&=w'+\psi\end{array}\right.
\end{equation}
allowing the representation of the strains from eq \eqref{eq:StrainEvaluatedSimple} using the generalized strains as:
\begin{equation}\label{eq:StrainEvaluatedSimpleGeneralized}
\left\{\begin{array}{l}
\epsilon_{xx}=\varepsilon+\rho_yy+\rho_zz\\
\epsilon_{xy}=\frac{1}{2}(\gamma_y-\varphi z)\\
\epsilon_{xz}=\frac{1}{2}(\gamma_z+\varphi y)
\end{array}\right.
\end{equation}
Now the stresses in equation \eqref{eq:ConstitutiveRelationshipEngineeringNotation} can be represented with the displacements as 
\begin{equation}\label{key}
\left\{\begin{array}{c}
\sigma_{xx}\\ \sigma_{xy}\\ \sigma_{xz}
\end{array}\right\} = \left\{\begin{array}{c}
E(u'-\theta'y+\psi'z)\\ G(v'-\phi'z-\theta)\\ G(w'+\phi'y+\psi)
\end{array}\right\} = \left\{\begin{array}{c}
E(\varepsilon+\rho_yy+\rho_zz)\\ G(\gamma_y-\varphi z)\\ G(\gamma_z+\varphi y)
\end{array}\right\}
\end{equation}
%Stress at the beam axis is expressed using Hooke's Law for linear isotropic homogeneous material
%\begin{equation}\label{LinearElasticStressStrainRelationship}
%\sigma_{ij}=\lambda\delta_{ij}\epsilon_{kk}+2\mu\epsilon_{ij}
%\end{equation}
%expanded out 
%\begin{equation}\label{StressEvaluated}
%\underline{\sigma}=\left[\begin{array}{ccc}
%(\lambda+2\mu)(u'-\theta'y+\psi'z)&\mu(v'-\phi'z-\theta)&\mu(w'+\phi'y+\psi)\\
%\mu(v'-\phi'z-\theta)&\lambda(u'-\theta'y+\psi'z)&0\\
%\mu(w'+\phi'y+\psi)&0&\lambda(u'-\theta'y+\psi'z)
%\end{array}\right]
%\end{equation}
%\begin{equation}\label{key}
%\sigma_{ij}=\lambda\delta_{ij}\epsilon_{k,k}+\mu\left(u_{i,j}+u_{j,i}\right)
%\end{equation}
\begin{equation}\label{eq:TotalVirtualStrainEnergyExpression}
U = \int_{\volume}\sigma_{ij}\epsilon_{ij}d\volume
\end{equation}
Now the inner product of the total virtual strain energy can be expanded
\begin{equation}\label{eq:TotalStrainEnergyExpanded}
U=\int_{\volume}\left[\sigma_{xx}\epsilon_{xx}+2\sigma_{xy}\epsilon_{xy}+2\sigma_{xz}\epsilon_{xz}\right]d\volume
\end{equation}
Then expand virtual strains into the generalized strains corresponding to the degrees of freedom of the beam element and collect terms on generalized strains
\begin{equation}\label{eq:TotalStrainEnergyGeneralized}
 U=\int_{\volume}\left[\sigma_{xx}\varepsilon+\sigma_{xx}y\rho_y+\sigma_{xx}z\rho_z+\sigma_{xy}\gamma_y+\sigma_{xz}\gamma_z+(\sigma_{xz}y-\sigma_{xy}z)\varphi\right]d\volume
\end{equation}
this internal mechanical energy expression allows us to recognize stresses conjugate with each generalized strain as the corresponding stress for that phenomena. Integration allows the determination of the forces and moments related to each generalized strain as
\begin{equation}\label{eq:generalizedForces}
\left\{\arraycolsep=1pt\begin{array}{rcl}
N=&\int_A\sigma_{xx}dA&=E( A\varepsilon+S_y\rho_y+S_z\rho_z)\\
M_y=&\int_A\sigma_{xx}zdA&=E(S_z\varepsilon+I_{xy}\rho_y+I_y\rho_z)\\
M_z=&\int_A\sigma_{xx}ydA&=E(S_y\varepsilon+I_z\rho_y+I_{xy}\rho_z)\\
Q_y=&\int_A\sigma_{xy}dA&=\kappa G(A\gamma_y-S_z\varphi)\\
Q_z=&\int_A\sigma_{xz}dA&=\kappa G(A\gamma_z+S_y\varphi)\\
M_x=&\int_A(\sigma_{xz}y-\sigma_{xy}z)dA&=\kappa G\left(A_y\gamma_z-A_z\gamma_y+J_x\varphi\right)
\end{array}\right.
\end{equation}
\begin{center}where, $ \left\{\begin{array}{l}
\ \kappa=\frac{6(1+\nu)}{7+6\nu}, \textit{for circular cross sections.}\\
\arraycolsep=.75pt\begin{array}{rlccccrl}
A&=\int_AdA&&&&&S_y&=\int_A ydA\\
S_z&=\int_AzdA&&&&&I_y&=\int_Az^2dA\\
I_z&=\int_Ay^2dA&&&&&J_x&=I_y+I_z
\end{array}
\end{array}\right. $\\\end{center}
$ \kappa $ is the shear coefficient which attempts to correct for the fact that the shear strain is not constant over the beam cross section. Assuming the central axis of the beam is coincident with the shear center, then $ A_y=A_z=I_{xy}=0 $. Which simplifies the conjugate forces to


\begin{equation}\label{eq:generalizedForcesSimplified}
\left\{\arraycolsep=1pt\begin{array}{rll}
N&=EA\varepsilon&=EAu'\\
M_y&=EI_y\rho_z&=EI_y\psi'\\
M_z&=EI_z\rho_y&=-EI_z\theta'\\
Q_y&=\kappa G A\gamma_y&=\kappa G A(v'-\theta)\\
Q_z&=\kappa G A\gamma_z&=\kappa G A(w'+\psi)\\
M_x&=\kappa G J_x\varphi&=\kappa G J_x\phi'
\end{array}\right.
\end{equation}

%\begin{equation}\label{eq:TotalVirtualStrainEnergyGeneralizedExpanded}
%\delta U=\int_{0}^{l}\left[N\delta\varepsilon+M_z\delta\rho_y+M_y\delta\rho_z+Q_y\delta\gamma_y+Q_z\delta\gamma_z+M_x\delta\varphi\right]dx
%\end{equation}
%upon expanding with forces defined in eq \eqref{eq:generalizedForcesSimplified}
%\begin{multline}
%\delta U=\int_{0}^{l}[EAu'\delta u'
%+EI_y\psi'\delta\psi'
%+EI_z\theta'\delta\theta'
%+\kappa G A(v'-\theta)(\delta v'-\delta\theta)\\
%+\kappa G A(w'+\psi)(\delta w'+\delta\psi)
%+\kappa G J_x\phi'\delta\phi']dx
%\end{multline}
%and using integration by parts to achieve a relation to the displacements 
%\begin{multline}\label{eq:TotalVirtualStrainEnergyDisplacementRel}
%\delta U=\int_{0}^{l}\{EAu''\delta u
%+\kappa G A(v''-\theta')\delta v
%+\kappa G A(w''+\psi')\delta w\\
%+[EI_y\psi''-\kappa G A(w'+\psi)]\delta\psi
%+[EI_z\theta''+\kappa G A(v'-\theta)]\delta\theta\\
%+\kappa G J_x\phi''\delta\phi\}dx + \delta U_b
%\end{multline}
%leaving $ \delta U_b $ as the boundary value terms to be neglected. The principle of virtual work for a quasistatic beam with no external work gives $ \delta U=0 $. Therefore, each term in the integrand of \eqref{eq:TotalVirtualStrainEnergyDisplacementRel} must equal zero.
\subsubsection{Differential Equations of Motion}
Now the equations are motion are derived for the Timoshenko beam element. External forces are not included in this derivation. Though they may easily be added to the diagram of figure \ref{fig:BeamDifferentialSection} and included in the analysis. It is also assumed that the cross section remains planar during deformation and the material properties are homogeneous through time and space. The derivation is the same as for a Euler-Bernoulli beam with the exception of the constitutive relations used at the end and the inclusion of torsion and axial degrees of freedom.
\begin{figure}
	\centering
	\def\svgwidth{600pt}
	\import{figures/}{BeamDifferentialSection.pdf_tex}
	\caption{Beam differential element with generalized forces.}
	\label{fig:BeamDifferentialSection}
\end{figure}
Using conservation of momentum and conservation of the moment of momentum a relationship between inertia and internal forces is developed. Applying summation of forces in the x-direction
\begin{equation}\label{key}
(N+\frac{1}{2}\frac{\partial N}{\partial x}dx)-(N-\frac{1}{2}\frac{\partial N}{\partial x}dx)=\rho A dx \frac{\partial^2u}{\partial x^2}
\end{equation}
by Simplifying the above equation, and performing the same steps for the other directions and moments we get
\begin{equation}\label{eq:EquilibriumEquations}
\left\{\arraycolsep=1pt\begin{array}{rl}
\frac{\partial N}{\partial x}&=\rho A\frac{\partial^2 u}{\partial x^2}\\
\frac{\partial Q_y}{\partial x}&=\rho A\frac{\partial^2 v}{\partial x^2}\\
\frac{\partial Q_z}{\partial x}&=\rho A\frac{\partial^2 w}{\partial x^2}\\
\frac{\partial M_y}{\partial x}-Q_z&=\rho I_y \frac{\partial^2\psi}{\partial x^2} + \rho J_x\Omega\frac{\partial\theta}{\partial x}\\
\frac{\partial M_z}{\partial x}+Q_y&=\rho I_z \frac{\partial^2\theta}{\partial x^2} - \rho J_x\Omega\frac{\partial\psi}{\partial x}\\
\frac{\partial M_x}{\partial x}&=\rho J_x\frac{\partial^2\phi}{\partial x^2}
\end{array}\right.
\end{equation}
The generalized forces of equation \eqref{eq:generalizedForcesSimplified} are substituted in the equilibrium equations \eqref{eq:EquilibriumEquations}
%\begin{equation}\label{eq:EquationsOfMotionIndependent}
%\left\{\arraycolsep=1pt\begin{array}{rl}
%EAu''&=\rho A\ddot{u}\\
%\kappa GA(v''-\theta')&=\rho A\ddot{v}\\
%\kappa GA(w''+\psi')&=\rho A\ddot{w}\\
%EI_y\psi''-\kappa GA(w'-\theta)&=\rho I_y\ddot{\psi}+\rho J_x\Omega\dot{\theta}\\
%EI_z\theta''+\kappa GA(v'+\psi)&=\rho I_z\ddot{\theta}-\rho J_x\Omega\dot{\psi}\\
%\kappa GJ_z\phi''&=\rho J_x\ddot{\phi}
%\end{array}\right.
%\end{equation}
\begin{subequations}\label{eq:EquationsOfMotionIndependent}
\begin{empheq}[left={\empheqlbrace\,}]{align}
EAu''&=\rho A\ddot{u}\label{eq:EquationsOfMotionIndependent_u}\\
\kappa GA(v''-\theta')&=\rho A\ddot{v}\label{eq:EquationsOfMotionIndependent_v}\\
\kappa GA(w''+\psi')&=\rho A\ddot{w}\label{eq:EquationsOfMotionIndependent_w}\\
EI_y\psi''-\kappa GA(w'-\theta)&=\rho I_y\ddot{\psi}+\rho J_x\Omega\dot{\theta}\label{eq:EquationsOfMotionIndependent_psi}\\
EI_z\theta''+\kappa GA(v'+\psi)&=\rho I_z\ddot{\theta}-\rho J_x\Omega\dot{\psi}\label{eq:EquationsOfMotionIndependent_theta}\\
\kappa GJ_z\phi''&=\rho J_x\ddot{\phi}\label{eq:EquationsOfMotionIndependent_phi}
\end{empheq}
\end{subequations}
In Matrix form, this system of equations can be represented by this equation
\begin{equation}\label{key}
\bunderline{\mathcal{M}}^e\ddot{\vec{\mathbf{u}}}+\bunderline{\mathcal{G}}^e\dot{\vec{\mathbf{u}}}-\frac{\partial}{\partial x}\bunderline{\mathcal{S}}^e\vec{\mathbf{u}}-\bunderline{\mathcal{P}}\bunderline{\mathcal{S}}^e\vec{\mathbf{u}}
\end{equation}
where,
\begin{equation}\label{key}
\def\cs{2em}
\def\csn{3em}
\newcommand{\SetToWidest}[1]{\makebox[\cs]{$#1$}}%
\left\{\def\arraystretch{1}\arraycolsep=1pt\begin{array}{@{}rlrl}
\bunderline{\mathcal{M}}^e&=\!\left[\def\arraystretch{.8}\arraycolsep=-.5pt\begin{array}{cccccc}
\SetToWidest{\rho A}&\SetToWidest{0}&\SetToWidest{0}&\SetToWidest{0}&\SetToWidest{0}&\SetToWidest{0}\\
0&\rho A&0&0&0&0\\
0&0&\rho I_y&0&0&0\\
0&0&0&\rho I_z&0&0\\
0&0&0&0&\rho A&0\\
0&0&0&0&0&\rho J_x
\end{array}\right]&\bunderline{\mathcal{G}}^e&=\!\left[\def\arraystretch{.8}\arraycolsep=-.5pt\begin{array}{cccccc}
\makebox[\cs]{0}&\makebox[\cs]{0}&\makebox[\cs]{0}&\makebox[\cs]{0}&\makebox[\cs]{0}&\makebox[\cs]{0}\\
0&0&0&0&0&0\\
0&0&0&\rho J_x\Omega&0&0\\
0&0&-\rho J_x\Omega&0&0&0\\
0&0&0&0&0&0\\
0&0&0&0&0&0
\end{array}\right]\\
&&\\[-1em]
\bunderline{\mathcal{P}}^e&=\!\left[\def\arraystretch{.8}\arraycolsep=-.5pt\begin{array}{cccccc}
\SetToWidest{0}&\SetToWidest{0}&\SetToWidest{0}&\SetToWidest{0}&\SetToWidest{0}&\SetToWidest{0}\\
0&0&0&0&0&0\\
0&-1&0&0&0&0\\
1&0&0&0&0&0\\
0&0&0&0&0&0\\
0&0&0&0&0&0
\end{array}\right]&\bunderline{\mathcal{S}}^e&=\!\left[\def\arraystretch{.8}\arraycolsep=-1pt\begin{array}{@{}cccccc@{}}
\makebox[\csn]{$\kappa GA \frac{\partial}{\partial x}$}&\makebox[\csn]{0}&\makebox[\csn]{0}&\makebox[\csn]{$-\kappa G \frac{\partial}{\partial x}$}&\makebox[\csn]{0}&\makebox[\csn]{0}\\
0&\kappa GA\frac{\partial}{\partial x}&\kappa GA&0&0&0\\
0&0&EI_y\frac{\partial}{\partial x}&0&0&0\\
0&0&0&EI_z\frac{\partial}{\partial x}&0&0\\
0&0&0&0&EA\frac{\partial}{\partial x}&0\\
0&0&0&0&0&\kappa GJ_x\frac{\partial}{\partial x}
\end{array}\right]
\end{array}\right.
\end{equation}
and $ \vec{\mathbf{u}}=[v,w,\psi,\theta,u,\phi]^{\T} $ The principle of virtual displacements is utilized on the equations of motion to obtain the weak for of the equations of motion and integrated over the length of the beam.
\begin{equation}\label{eq:BeamDiffEquationVirtualStart}
\int_0^l \delta\vec{\mathbf{u}}^\T\bunderline{\mathcal{M}}^e\ddot{\vec{\mathbf{u}}}dx+\int_0^l \delta\vec{\mathbf{u}}^\T\bunderline{\mathcal{G}}^e\dot{\vec{\mathbf{u}}}-\int_0^l\delta\vec{\mathbf{u}}^\T\frac{\partial}{\partial x}\bunderline{\mathcal{S}}^e\vec{\mathbf{u}}dx-\int_0^l\delta\vec{\mathbf{u}}^\T\bunderline{\mathcal{P}}\bunderline{\mathcal{S}}^e\vec{\mathbf{u}}dx=0
\end{equation}
integration by parts on the third term and replacing $ \underline{\mathcal{S}} $ with $ \bunderline{\mathcal{D}}^e\bunderline{\mathcal{B}} $, and making use of the Identity matrix, $ \bunderline[2]{\mathbf{I}} $
\begin{equation}\label{key}
\int_0^l \delta\vec{\mathbf{u}}^\T\bunderline{\mathcal{M}}^e\ddot{\vec{\mathbf{u}}}dx+\int_0^l \delta\vec{\mathbf{u}}^\T\bunderline{\mathcal{G}}^e\dot{\vec{\mathbf{u}}}+\int_0^l\delta\vec{\mathbf{u}}^\T(\frac{\partial}{\partial x}\bunderline[2]{\mathbf{I}}-\bunderline{\mathcal{P}})\bunderline{\mathcal{D}}^e\bunderline{\mathcal{B}}\vec{\mathbf{u}}dx=0
\end{equation}
\begin{equation}\label{key}
\def\cs{1.5em}
\bunderline{\mathcal{D}}^e=\left[\def\arraystretch{.8}\arraycolsep=1.8pt\begin{array}{cccccc}
\makebox[\cs]{$\kappa GA$}&\makebox[\cs]{0}&\makebox[\cs]{0}&\makebox[\cs]{0}&\makebox[\cs]{0}&\makebox[\cs]{0}\\
0&\kappa GA&0&0&0&0\\
0&0&EI_y&0&0&0\\
0&0&0&EI_z&0&0\\
0&0&0&0&EA&0\\
0&0&0&0&0&\kappa GJ_x
\end{array}\right] \&\quad \bunderline{\mathcal{B}}=\left[\def\arraystretch{.8}\arraycolsep=1.8pt\begin{array}{cccccc}
\makebox[\cs]{$\frac{\partial}{\partial x}$}&\makebox[\cs]{0}&\makebox[\cs]{0}&\makebox[\cs]{-1}&\makebox[\cs]{0}&\makebox[\cs]{0}\\
0&\frac{\partial}{\partial x}&1&0&0&0\\
0&0&\frac{\partial}{\partial x}&0&0&0\\
0&0&0&\frac{\partial}{\partial x}&0&0\\
0&0&0&0&\frac{\partial}{\partial x}&0\\
0&0&0&0&0&\frac{\partial}{\partial x}
\end{array}\right]
\end{equation}
Notice that $ \frac{\partial}{\partial x}\bunderline{I}-\bunderline{\mathcal{P}}=\bunderline{\mathcal{B}}^\T $ so the equation of motion becomes
\begin{equation}\label{eq:EquationOfMotionVirtual}
\int_0^l \delta\vec{\mathbf{u}}^\T\bunderline{\mathcal{M}}^e\ddot{\vec{\mathbf{u}}}dx+\int_0^l \delta\vec{\mathbf{u}}^\T\bunderline{\mathcal{G}}^e\dot{\vec{\mathbf{u}}}+\int_0^l\delta\vec{\mathbf{u}}^\T\bunderline{\mathcal{B}}^\T\bunderline{\mathcal{D}}^e\bunderline{\mathcal{B}}\vec{\mathbf{u}}dx=0
\end{equation}
Now the solution of this differential system motivates a separation of variables that will be discussed in the next section.
\subsubsection{Shape Functions}
The displacements thus far have been assumed to be functions of both position and time. Now the total displacement is seperated into functions that depend on time and functions that depend on position. This is a fundamental part of the discretization of the beam element, and the use the finite element method. 
\begin{equation} \label{eq:FEGoverning}
\left\{\begin{array}{rl}
\vec{\mathbf{u}}(x,t)&=\bunderline{\mathbf{N}}(x)\vec{\mathbf{q}}(t)\\
\dot{\vec{\mathbf{u}}}(x,t)&=\bunderline{\mathbf{N}}(x)\dot{\vec{\mathbf{q}}}(t)\\
\ddot{\vec{\mathbf{u}}}(x,t)&=\bunderline{\mathbf{N}}(x)\ddot{\vec{\mathbf{q}}}(t)\\
\delta\vec{\mathbf{u}}(x,t)&=\bunderline{\mathbf{N}}(x)\delta\vec{\mathbf{q}}(t)
\end{array}\right.
\end{equation}
where, $ \vec{\mathbf{q}}=[v_1,w_1,\psi_1,\theta_1,v_2,w_2,\psi_2,\theta_2,u_1,\phi_1,u_2,\phi_2]^\T $. The specfic order of $ \vec{\mathbf{q}} $ is chosen with $ u $ and $ \phi $ at the end to ease the condensation of the axial and torsional degrees of freedom out of the system if their use is not necessary for the system of interest. The shape functions $ \bunderline{\mathbf{N}}(x) $ must solve the static portion of the differential equations \eqref{eq:EquationsOfMotionIndependent}.
%\begin{subequations}\label{eq:EulerLagrangianEquations}
%	\begin{empheq}[left={\empheqlbrace\,}]{align}
%	&EAu''=0 \label{eq:EulerLagrangianAxial}\\
%	&\kappa GA(v''-\theta')=0 \label{eq:EulerLagrangianSheary}\\
%	&\kappa GA(w''+\psi')=0 \label{eq:EulerLagrangianShearz}\\
%	&\tilde{E}I_y\psi''-\kappa G A(w'+\psi)=0 \label{eq:EulerLagrangianBendingy}\\
%	&\tilde{E}I_z\theta''+\kappa G A(v'-\theta)=0 \label{eq:EulerLagrangianBendingz}\\
%	&\kappa G J_x\phi''=0 \label{eq:EulerLagrangianTorsion}
%	\end{empheq}
%\end{subequations}
%These are the governing differential equations for a Timoshenko beam element.

\begin{figure}
	\centering
	\def\svgwidth{400pt}
	\import{figures/}{BeamElement.pdf_tex}
	\caption{Beam Element with nodal displacements.}
	\label{fig:BeamElem}
\end{figure}
These shape functions are chosen as a polynomials that satisfy the boundary nodal displacements and rotations at the ends of a beam element. These nodal degrees of freedom, depicted in Figure \ref{fig:BeamElem} are considered to be interpolated through the beam element by the shape functions. Interpolation functions chosen are listed in Equation \eqref{eq:DisplacementInterpolationFunctionsChosen}. Axial displacement, $ u $, and torsional rotation, $ \phi $ are independent, so their shape functions are chosen as polynomials that satisfy the differential equation. Conversely, transverse displacements, $ v $ \& $ w $, and bending rotations, $ \psi $ \& $ \theta $ are coupled. Coupling of the shape functions has been proven to reduce some negative effects of linearly interpolated elements \cite{luo2008efficient}. Polynomial functions are chosen for $ v $ \& $ w $ and their rotational counterparts are derived using the differential relations.
\begin{subequations}\label{eq:DisplacementInterpolationFunctionsChosen}
\begin{empheq}[left={\empheqlbrace\,}]{align}
u&=c_1+c_2x \label{eq:AxialInterpolationFunction}\\
v&=c_3+c_4x+c_5x^2+c_6x^3 \label{eq:TransverseyInterpolationFunction}\\ 
w&=c_7+c_8x+c_9x^2+c_{10}x^3 \label{eq:TransversezInterpolationFunction}\\
\phi&=c_{11}+c_{12}x \label{eq:TorsionInterpolationFunction}
\end{empheq}
\end{subequations}
$ c_{1,2,...} $ are the unknown constants of the polynomial solutions. Using transverse displacement of equations \eqref{eq:TransverseyInterpolationFunction} \& \eqref{eq:TransversezInterpolationFunction} in the differential equations \eqref{eq:EquationsOfMotionIndependent_v}, \eqref{eq:EquationsOfMotionIndependent_w}, \eqref{eq:EquationsOfMotionIndependent_psi}, \eqref{eq:EquationsOfMotionIndependent_theta} the interpolation functions of bending rotations are derived as:
\begin{subequations}\label{eq:DisplacmentInterpolationFunctionsDerived}
\begin{empheq}[left={\empheqlbrace\,}]{align}
\psi&=K_yc_{10}-c_8-2c_9x-3c_{10}x^2 \label{eq:RotationyInterpolationFunction}\\
\theta&=K_zc_6+c_4+2c_5x+3c_{6}x^2 \label{eq:RotationzInterpolationFunction}
\end{empheq}
\end{subequations}
where, $ K_y=\frac{6EI_y}{\kappa GA} $\& $ K_z=\frac{6EI_z}{\kappa GA} $
Boundary Conditions
Boundary conditions for the interpolation polynomials of equations \eqref{eq:DisplacementInterpolationFunctionsChosen} \& \eqref{eq:DisplacmentInterpolationFunctionsDerived} are defined as the components of the vector $ \vec{\mathbf{q}} $$ u_j=u(x_j) $ and similarily for other degrees of freedom. Where, $ j=1,2 $ and defines the two states. In this derivation, $ x_1=0 $ and $ x_2=l $. Application of these boundary condition results in this relation between the polynomial constants and the boundary conditions.
\begin{equation}\label{DisplacementInterpolationBoundaryMatrix}
\def\cs{4em}
\newcommand{\WidestEntry}{$\scriptstyle K_y\!-\!3l^2$}%
%\newcommand{\SetToWidest}[1]{\makebox[\widthof{\WidestEntry}]{$#1$}}%
\newcommand{\SetToWidest}[1]{\makebox[\cs]{$#1$}}%
\left\{\def\arraystretch{.8}\begin{array}{@{}c@{}}
u_1\\u_2\\v_1\\v_2\\w_1\\w_2\\\psi_1\\\psi_2\\\theta_1\\\theta_2\\\phi_1\\\phi_2
\end{array}\right\}\hspace{-4pt}=\hspace{-4pt}\left[\arraycolsep=-.68em\def\arraystretch{.8}\begin{array}{@{}lccccccccccr@{}}
\makebox[\cs/2][l]{1}& \SetToWidest{0}& \SetToWidest{0}& \SetToWidest{0}& \SetToWidest{0}& \SetToWidest{0}& \SetToWidest{0}& \SetToWidest{0}& \SetToWidest{0}& \SetToWidest{0}& \SetToWidest{0}& \makebox[\cs/2][r]{0}\\
1& l& 0& 0& 0& 0& 0& 0& 0& 0& 0& 0\\
0& 0& 1& 0& 0& 0& 0& 0& 0& 0& 0& 0\\
0& 0& 1& l& l^2& l^3& 0& 0& 0& 0& 0& 0\\
0& 0& 0& 0& 0& 0& 1& 0& 0& 0& 0& 0\\
0& 0& 0& 0& 0& 0& 1& l& l^2& l^3& 0& 0\\
0& 0& 0& 0& 0& 0& 0& \text{-}1& 0& \text{-}K_y& 0& 0\\
0& 0& 0& 0& 0& 0& 0& \text{-}1& \text{-}2l& \text{-}K_y\text{-}3l^2& 0& 0\\
0& 0& 0& 1& 0&  K_z& 0& 0& 0& 0& 0& 0\\
0& 0& 0& 1& 2l&  K_z\!\text{+}3l^2& 0& 0& 0& 0& 0& 0\\
0& 0& 0& 0& 0& 0& 0& 0& 0& 0& 1& 0\\
0& 0& 0& 0& 0& 0& 0& 0& 0& 0& 1& l
\end{array}\right]\hspace{-6pt}\left\{\def\arraystretch{.8}\begin{array}{@{}c@{}}
c_1\\c_2\\c_3\\c_4\\c_5\\c_6\\c_7\\c_8\\c_9\\c_{10}\\c_{11}\\c_{12}
\end{array}\right\}
\end{equation}
Inversion of this matrix results in a system of equations defining the constant $ c_1 $ through $ c_{12} $. These constants are then substituted in to the polynomial expressions \eqref{eq:DisplacementInterpolationFunctionsChosen} \& \eqref{eq:DisplacmentInterpolationFunctionsDerived} giving the interpolations as functions of the nodal displacements.
\begin{equation}\label{eq:InterpolationFunctions}
\left\{\begin{array}{l}
u=N_1u_1+N_2u_2\\
v=T_{t_1y}v_1+T_{t_2y}v_2+T_{r_1y}\theta_1+T_{r_2y}\theta_2\\
w=T_{t_1z}w_1+T_{t_2w}w_2+T_{r_1z}\psi_1+T_{r_2z}\psi_2\\
\psi=R_{t_1z}w_1+R_{t_2w}w_2+R_{r_1z}\psi_1+R_{r_2z}\psi_2\\
\theta=R_{t_1y}v_1+R_{t_2y}v_2+R_{r_1y}\theta_1+R_{r_2y}\theta_2\\
\phi=N_1\phi_1+N_2\phi_2
\end{array}\right.
\end{equation}
with;
\begin{equation}\label{eq:TimoShapeFunctions}
\left\{\arraycolsep=.2em\begin{array}{@{}ll}
N_1=1-\zeta&N_2=\zeta\\
T_{t_1y,z}=\frac{1}{1+\alpha_{y,z}}(2\zeta^3-3\zeta^2-\alpha_{y,z}\zeta+1+\alpha_{y,z})&T_{t_2y,z}=\frac{1}{1+\alpha_{y,z}}(-2\zeta^3+3\zeta^2+\alpha_{y,z}\zeta)\\
T_{r_1y,z}=\frac{l}{1+\alpha_{y,z}}[\zeta^3-(2+\frac{1}{2}\alpha_{y,z})\zeta^2+(1+\frac{1}{2}\alpha_{y,z})\zeta]&T_{r_2y,z}=\frac{l}{1+\alpha_{y,z}}[\zeta^3-(1-\frac{1}{2}\alpha_{y,z})\zeta^2-\frac{1}{2}\alpha_{y,z}\zeta]\\
R_{t_1y,z}=\frac{6/l}{1+\alpha_{y,z}}(\zeta^2-\zeta)&R_{t_2y,z}=\frac{6/l}{1+\alpha_{y,z}}(-\zeta^2+\zeta)\\
R_{r_1y,z}=\frac{1}{1+\alpha_{y,z}}(3\zeta^2-(4+\alpha_{y,z})\zeta+1+\alpha_{y,z})&R_{r_2y,z}=\frac{1}{1+\alpha_{y,z}}(3\zeta^2-(2-\alpha_{y,z})\zeta)
\end{array}\right.
\end{equation}
where, $ \alpha_y=2K_y/l^2=\frac{12EI_y}{\kappa GAl^2} $, $ \alpha_z=2K_z/l^2=\frac{12EI_z}{\kappa GAl^2} $, \& $ \zeta=x/l $.
\eqref{eq:InterpolationFunctions} is expressed in matrix form as it appears in \eqref{eq:FEGoverning} where
\begin{equation}\label{ShapeFunctionMatrix}
\def\cs{2em}
\bunderline{\mathbf{N}}(x)=\left[\def\arraystretch{.8}\arraycolsep=0em\begin{array}{cccccccccccc}
\makebox[\cs]{$T_{t_1y}$}&\makebox[\cs]{0}&\makebox[\cs]{0}&\makebox[\cs]{$T_{r_1y}$}&\makebox[\cs]{$T_{t_2y}$}&\makebox[\cs]{0}&\makebox[\cs]{0}&\makebox[\cs]{$T_{r_2y}$}&\makebox[\cs]{0}&\makebox[\cs]{0}&\makebox[\cs]{0}&\makebox[\cs]{0}\\
0&T_{t_1z}&-T_{r_1z}&0&0&T_{t_2z}&-T_{r_2z}&0&0&0&0&0\\
0&R_{t_1z}&R_{r_1z}&0&0&R_{t_2z}&R_{r_2z}&0&0&0&0&0\\
-R_{t_1y}&0&0&R_{r_1y}&-R_{t_2y}&0&0&R_{r_2y}&0&0&0&0\\
0&0&0&0&0&0&0&0&N_1&0&N_2&0\\
0&0&0&0&0&0&0&0&0&N_1&0&N_2
\end{array}\right]
\end{equation}
with the generalized displacement vector, $ \vec{\mathbf{q}}=[v_1,w_1,\psi_1,\theta_1,v_2,w_2,\psi_2,\theta_2,u_1,\phi_1,u_2,\phi_2]^\T $\par
Shape functions depend on the term $ \alpha $ which is sometimes called the shear correction factor. This shear correction factor is proportional to the square of the ratio of radius to length of the beam element. So, as the length increases relative to the radius, $ \alpha $ tends to zero. It will be evident in the following section that as $ \alpha $ approaches zero, the equations of motion approach the equations of the Bernoulli-Euler beam. A spatial representation of the shape functions of equation \eqref{eq:TimoShapeFunctions} is given in figure \ref{fig:ShapeFunctions}.
\begin{figure}[h!]	
	\begin{subfigure}{.5\linewidth}
		\tikzset{every picture/.style={scale=1},every axis/.style={title style={yshift=-.8em}}}%, every axis/.style={hide axis}}%
		\centering
		\import{figures/}{Shapes.tex}
		\caption{length to radius ratio of 100.}
	\end{subfigure}
	\begin{subfigure}{.5\linewidth}
		\tikzset{every picture/.style={scale=1},every axis/.style={title style={yshift=-.8em}}}%, every axis/.style={hide axis}}%
		\centering
		\import{figures/}{Shapes2.tex}
		\caption{length to radius ratio of 1.}
	\end{subfigure}
	\caption{Shape Functions as they vary with $ \zeta $ using two different ratios of length to radius of beam element.}
	\label{fig:ShapeFunctions}
\end{figure}
\subsubsection{Finite Equations of Motion}
Substitute \eqref{eq:FEGoverning} into \eqref{eq:EquationOfMotionVirtual}
\begin{equation}\label{key}
\int_0^l \bunderline{\mathbf{N}}^\T\delta\vec{\mathbf{q}}^\T\bunderline{\mathcal{M}}^e\bunderline{\mathbf{N}}\ddot{\vec{\mathbf{q}}}dx+\int_0^l \bunderline{\mathbf{N}}^\T\delta\vec{\mathbf{q}}^\T\bunderline{\mathcal{G}}^e\bunderline{\mathbf{N}}\dot{\vec{\mathbf{q}}}dx+\int_0^l\bunderline{\mathbf{N}}^\T\delta\vec{\mathbf{q}}^\T\bunderline{\mathcal{B}}^\T\bunderline{\mathcal{D}}^e\bunderline{\mathcal{B}}\bunderline{\mathbf{N}}\vec{\mathbf{q}}dx=0
\end{equation}
note that now $ \vec{\mathbf{q}} $ is not dependent on x so it, and it's derivatives, may be pulled out of the integral. Also, a substitution of $ \bunderline{\mathbf{B}}=\bunderline{\mathcal{B}}\bunderline{\mathbf{N}} $ is made leading to the differential equation
\begin{equation}\label{eq:GoverningDifferentialEquationDiscrete}
\int_0^l \bunderline{\mathbf{N}}^\T\bunderline{\mathcal{M}}^e\bunderline{\mathbf{N}}dx\ddot{\vec{\mathbf{q}}}+\int_0^l \bunderline{\mathbf{N}}^\T\bunderline{\mathcal{G}}^e\bunderline{\mathbf{N}}dx\dot{\vec{\mathbf{q}}}+\int_0^l\bunderline{\mathbf{B}}^\T\bunderline{\mathcal{D}}^e\bunderline{\mathbf{B}}dx\vec{\mathbf{q}}=0
\end{equation}
Define 
\begin{subequations}\label{key}
\begin{empheq}[left={\empheqlbrace\,}]{align}
\bunderline{\mathbf{M}}^e&=\int_0^l \bunderline{\mathbf{N}}^\T\bunderline{\mathcal{M}}^e\bunderline{\mathbf{N}}dx\label{eq:ConsistentMassMatrix}\\
\bunderline{\mathbf{G}}^e&=\int_0^l \bunderline{\mathbf{N}}^\T\bunderline{\mathcal{G}}^e\bunderline{\mathbf{N}}dx\label{eq:ConsistentGyroMatrix}\\
\bunderline{\mathbf{K}}^e&=\int_0^l\bunderline{\mathbf{B}}^\T\bunderline{\mathcal{D}}^e\bunderline{\mathbf{B}}dx\label{eq:ConsistentStiffnessMatrix}
\end{empheq}
\end{subequations}
So that, the general equations of motion for the timoshenko beam element are
\begin{equation}\label{eq:EquationsOfMotionTimoElementGeneral}
\bunderline{\mathbf{M}}^e\ddot{\vec{\mathbf{q}}}+\bunderline{\mathbf{G}}^e\dot{\vec{\mathbf{q}}}+\bunderline{\mathbf{K}}^e\vec{\mathbf{q}}=0
\end{equation}
\subsubsection{Rotating Internal Damping}
Rotating damping is the main cause of instability in rotating machines. Non-rotating damping, such as the damping contributions from bearing supports, introduce a stabilizing effect. But, as rotating damping is dependent on rotation, its direction of force can contribute to destabilization. Typically, friction components such as bearings with shrink fits or oil bearings are responsible for this destabilizing force. Due to the inherent complexity of modeling loose bearing components, or shrink fit dynamics, the analysis of the internal damping in the shaft elements is considered alone. This will allow for the study of the destabilizing effect in general, and the factors that may contribute stability such as structural damping and anisotropy of supports. Another area of interest is the design of components for specific rotor geometry to maximize the stability in the system. This stability analysis is only possible with the inclusion of some destabilizing force, \cite{genta2007dynamics},\cite{genta2004persistent},\cite{kandil2005rotor},\cite{zorzi1977finite}.\par 
To motivate understanding of this force a simple derivation is provided with a rotating damping whose force is proportional to the flex of rate of change of the flex of the shaft. Obviously this is easier to define in the rotating reference frame, as the variables in this coordinate system directly represent the flex of the shaft from its neutral position. This relationship is as follows:
\begin{equation}\label{eq:LinearViscousDampingRot}
\vec{\mathcal{F}}_{\xi\eta}=-c_r
\left\{\begin{array}{c}
\dot{\xi}_c\\
\dot{\eta}_c
\end{array}\right\}
\end{equation}
Now to translate this force back to the stationary reference frame, we will see the rotation transformation matrix 
\begin{equation}\label{eq:RotationTransformation2D}
\bunderline{\mathcal{R}}=\left[\begin{array}{cc}
\cos{\Omega t}& \sin{\Omega t}\\
-\sin{\Omega t}& \cos{\Omega t}
\end{array}\right]
\end{equation}
transforms stationary into rotating coordinates 
\begin{equation}\label{eq:LinRotTransformation}
\left\{\begin{array}{rl}
\left\{\begin{array}{c}
\xi_c\\
\eta_c
\end{array}\right\}&=\bunderline{\mathcal{R}}\left\{\begin{array}{c}
y_c\\
z_c
\end{array}\right\}\\
&\\[-1em]
\left\{\begin{array}{c}
\dot{\xi}_c\\
\dot{\eta}_c
\end{array}\right\}&=\bunderline{\mathcal{R}}\left\{\begin{array}{c}
\dot{y}_c\\
\dot{z}_c
\end{array}\right\}+\dot{\bunderline{\mathcal{R}}}\left\{\begin{array}{c}
y_c\\
z_c
\end{array}\right\}
\end{array}\right.
\end{equation}
where, \begin{equation}
\dot{\bunderline{\mathcal{R}}}=\Omega\left[\begin{array}{cc}
-\sin{\Omega t}& \cos{\Omega t}\\
-\cos{\Omega t}& -\sin{\Omega t}
\end{array}\right]
\end{equation}
substituting the second equation in \ref{eq:LinRotTransformation} for the velocities in \ref{eq:LinearViscousDampingRot}
\begin{equation}\label{eq:LinearViscousDampingTrans}
\vec{\mathcal{F}}_{xy}=-c_r\left\{\begin{array}{c}
\dot{y}_c\\
\dot{z}_c
\end{array}\right\}-c_r\Omega\left[\begin{array}{@{}rc}
0&1\\
-1&0
\end{array}\right]\left\{\begin{array}{c}
y_c\\
z_c
\end{array}\right\}
\end{equation} 
From the equation \eqref{eq:LinearViscousDampingTrans} we see a dependence on both velocity and position. The portion dependent on the velocity is inherently stable as pulls opposite the motion. Conversely, the portion dependent on position cross couples the two displacements. This causes a destabilizing effect that grows as $ \Omega $ increases.
For the beam element, the constitutive relationship is \cite{zorzi1977finite}
\begin{equation}\label{eq:LinearViscousHystericConstitutiveRelation}
\sigma_{xx}=E\left\{\frac{\epsilon_{xx}}{\sqrt{1+\eta_h^2}}+\left(\eta_v+\frac{\eta_h}{\omega\sqrt{1+\eta_h^2}}\right)\dot{\epsilon}_{xx}\right\}
\end{equation}
through the use of kinematic relations to obtain strain displacement relations,
\begin{equation}\label{eq:straindisplacement}
\left\{\begin{array}{rl}
\epsilon_{xx}&=-r\cos{\left(\Omega-\omega \right)t}\frac{\partial^2 R}{\partial x^2}\\
\dot{\epsilon}_{xx}&=(\Omega-\omega)r\sin{(\Omega-\omega)t}\frac{\partial^2 R}{\partial x^2} - r\cos{(\Omega-\omega)t}\frac{\partial}{\partial t}\frac{\partial^2 R}{\partial x^2}
\end{array}\right.
\end{equation}
and inspection to obtain moment equations,
\begin{equation}\label{eq:momentequations}
\left\{\begin{array}{rl}
M_y &= \int_0^{2\pi}\int_0^a[w+r\sin{\Omega t}]\sigma_{xx}dr(rd(\Omega t))\\
M_z &= \int_0^{2\pi}\int_0^a-[v+r\cos{\Omega t}]\sigma_{xx}dr(rd(\Omega t))
\end{array}\right.
\end{equation}
leads to completing the integral pf \eqref{eq:momentequations} 
\begin{equation}\label{key}
\def\cs{8em}
\def\cd{1.5em}
\left\{\begin{array}{@{}c@{}}
M_y\\
M_z
\end{array}\right\} = E I\left[\def\arraystretch{.8}\arraycolsep=-1em\begin{array}{@{}cc@{}}
\makebox[\cs]{$\frac{1+\eta_h}{\sqrt{1+\eta_h^2}}$}&\makebox[\cs]{$\left(\frac{\eta_h}{\sqrt{1+\eta_h^2}}+\eta_h\Omega\right)$}\\
\left(\frac{\eta_h}{\sqrt{1+\eta_h^2}}+\eta_h\Omega\right)&-\frac{1+\eta_h}{\sqrt{1+\eta_h^2}}
\end{array}\right]\left\{\arraycolsep=0pt\begin{array}{@{}c@{}}
v''\\
w''
\end{array}\right\}+EI\left[\arraycolsep=0pt\begin{array}{@{}cc@{}}
\makebox[\cd]{$\eta_v$}&\makebox[\cd]{0}\\
0&-\eta_v
\end{array}\right]\left\{\arraycolsep=0pt\begin{array}{@{}c@{}}
\dot{v}''\\
\dot{w}''
\end{array}\right\}
\end{equation}
Now use the same strategy followed when solving for the weak form of the beam differential equations using the Principle of Virtual Displacements starting at equation \eqref{eq:BeamDiffEquationVirtualStart}. Then use the seperation of variables of defined by equation \eqref{eq:FEGoverning} to arrive at the total beam element including internal damping as
\begin{equation}\label{TotalBeamFiniteElementEquationofMotion}
\bunderline{\mathbf{M}}^e\ddot{\vec{\mathbf{q}}}+(\eta_v\bunderline{\mathbf{K}}^e+\bunderline{\mathbf{G}}^e)\dot{\vec{\mathbf{q}}}+(\eta_a\bunderline{\mathbf{K}}^e+\eta_b\bunderline{\mathbf{C}}^e)\vec{\mathbf{q}}=0
\end{equation}
where,
\begin{equation}\label{key}
\left\{\begin{array}{cc}
\bunderline{\mathcal{I}}=\left[\begin{array}{cccccc}
0&1&0&0&0&0\\
-1&0&0&0&0&0\\
0&0&0&1&0&0\\
0&0&-1&0&0&0\\
0&0&0&0&0&0\\
0&0&0&0&0&0
\end{array}\right]&\begin{array}{c}
\bunderline{\mathbf{C}}^e=\int_0^l\bunderline{\mathbf{B}}^\T\bunderline{\mathcal{I}}\bunderline{\mathcal{D}}\bunderline{\mathbf{B}}dx\\
\eta_a=\frac{1+\eta_h}{\sqrt{1+\eta_h^2}}\\
\eta_b=\Omega\eta_v+\frac{\eta_h}{\sqrt{1+\eta_h^2}}
\end{array}
\end{array}\right.
\end{equation}
\lstinputlisting[language=Matlab]{code/TBeamStiff.m}
\lstinputlisting[language=Matlab]{code/TBeamDamp.m}
\lstinputlisting[language=Matlab]{code/TBeamMass.m}
\section{Disk Nodal Equations}
Since the beam element has been discretized into nodal degrees of freedom, so long as the locations of disks in the model are chosen to coincide with one of these nodal locations, the expressions for stiffness and inertia can be directly combined with the global matrices at that node. The mass element, in this work, is considered as a body at a point with inertia, gyroscopic moments, and unbalance considered as external forces.
\begin{equation}\label{eq:DiskNodeEquationofMotion}
\bunderline{\mathbf{M}}^d\ddot{\vec{\mathbf{q}}}_k+\bunderline{\mathbf{G}}^d\dot{\vec{\mathbf{q}}}_k=\vec{\mathbf{F}}^d
\end{equation}

the superscript $ d $ represents that the matrix or array is for a disk, and the subscript $ _k $ on the displacement array indicates the array is only displacements for a single node, written out as: $ \vec{\mathbf{q}}_k = [v,w,\psi,\theta,u,\phi]^\T $. The matrices and forcing array of \eqref{eq:DiskNodeEquationofMotion} are as follows,
\begin{equation}\label{eq:DiskNodeEquations}
\def\cs{3em}
\left\{
\begin{array}{@{}c}
\arraycolsep=1pt\begin{array}{rlrl}
\bunderline{\mathbf{M}}^d & =\!\left[\def\arraystretch{.8}\arraycolsep=-.5em\begin{array}{cccccc}
\makebox[\cs]{$\rho Al$}&\makebox[\cs]{0}&\makebox[\cs]{0}&\makebox[\cs]{0}&\makebox[\cs]{0}&\makebox[\cs]{0}\\
0&\rho Al&0&0&0&0\\
0&0&\rho I_{z}&0&0&0\\
0&0&0&\rho I_{y}&0&0\\
0&0&0&0&\rho Al&0\\
0&0&0&0&0&\rho J_x
\end{array}\right] & \bunderline{\mathbf{G}} & = \!\left[\def\arraystretch{.8}\arraycolsep=-.5em\begin{array}{cccccc}
\makebox[\cs]{0}&\makebox[\cs]{0}&\makebox[\cs]{0}&\makebox[\cs]{0}&\makebox[\cs]{0}&\makebox[\cs]{0}\\
0&0&0&0&0&0\\
0&0&0&\rho J_x\Omega&0&0\\
0&0&-\rho J_x\Omega&0&0&0\\
0&0&0&0&0&0\\
0&0&0&0&0&0
\end{array}\right]\\
\end{array}\\
\\[-1em]
\vec{\mathbf{F}}^d=\left\{\def\arraystretch{.8}\begin{array}{c}
\rho Al\Omega^2\mathbf{e}\cos(\Omega t + \delta_\varepsilon)\\
\rho Al\Omega^2\mathbf{e}\sin(\Omega t + \delta_\varepsilon)\\
-\rho( I_y-J_x)\chi\sin(\Omega t + \delta_\varepsilon)\\
\rho( I_z-J_x)\chi\cos(\Omega t + \delta_\varepsilon)\\
0\\
0
\end{array}\right\}
\end{array}\right.
\end{equation}
\begin{figure}
	\centering
	\includegraphics[width=.5\linewidth]{./figures/Images/Figure_1b}
	\caption{Depiction of skew angle $\chi$}
	\label{fig:ChiAngleDepiction}
	\centering
\end{figure}
\par
 Disk unbalance is caused by an eccentricity, or a geometrical distance between the axis of rotation and the center of mass of the disk. Eccentricity is represented here as $\mathbf{e}$ and is equivalent to that geometric distance. The moment unbalance forces, the third and fourth equations of $ \vec{\mathbf{F}}^d $ in \eqref{eq:DiskNodeEquations}, are caused by the skew angle, $ \chi $, which is the angle the disk major axis forms with the axis of rotation as in Figure \ref{fig:ChiAngleDepiction}. The major axis is the axis normal from the disk face from which the polar moment of inertia, $ J_x $ is defined.
\lstinputlisting[language=Matlab]{code/DiskMass.m}
\lstinputlisting[language=Matlab]{code/DiskGyro.m}
\section{Bearing Nodal Equations}
Bearings in this work are to be considered massless points of stiffness and damping acting at a node. Represented by the local equations of motion
\begin{equation}\label{BearingNodeEquationsofMotion}
\bunderline{\mathbf{D}}^b\dot{\vec{\mathbf{q}}}_k+\bunderline{\mathbf{K}}^b\vec{\mathbf{q}}_k=0
\end{equation}
The superscript $ ^b $ indicates the matrix is for a bearing. A simple model for the stiffness and damping is used. Structural damping of the bearing is considered to be proportional to the stiffness, Raleigh Damping for the local bearing system. The stiffness matrix is comprised of only transverse stiffness terms, as 
\begin{equation}
\def\cs{2em}
\begin{array}{cc}
\bunderline{\mathbf{K}}^b=\left[\def\arraystretch{.8}\arraycolsep=0pt\begin{array}{cccccc}
\makebox[\cs]{$K_{yy}$}&\makebox[\cs]{$K_{yz}$}&\makebox[\cs]{0}&\makebox[\cs]{0}&\makebox[\cs]{0}&\makebox[\cs]{0}\\
K_{zy}&K_{zz}&0&0&0&0\\
0&0&0&0&0&0\\
0&0&0&0&0&0\\
0&0&0&0&0&0\\
0&0&0&0&0&0
\end{array}\right] & \bunderline{\mathbf{C}}^b=a\bunderline{\mathbf{K}}^b
\end{array}
\end{equation}
The stiffness is typically simplified further to represent an orthotropic bearing, where $ K_{yz}=K_{zy}=0 $ or further yet as a isotropic bearing, where $ K_{yz}=K_{zy}=0 $ \& $ K_{yy}=K_{zz}=K $. Generally, these unknown parameters of stiffness are determined by changing the values to achieve the correct natural frequencies of the system.
\lstinputlisting[language=Matlab]{code/BearingStiff.m}
\lstinputlisting[language=Matlab]{code/BearingDamp.m}
\section{Assembly of the Global Systems of Equations}
The matrices in the global system of equations are determined using the direct approach of taking the summation of the inertia, damping, stiffness, or force at each degree of freedom. This leaves the global system as
\begin{equation}\label{key}
\bunderline{\mathbf{M}}\ddot{\vec{\mathbf{q}}}+\bunderline{\mathbf{G}}\dot{\vec{\mathbf{q}}}+\bunderline{\mathbf{K}}\vec{\mathbf{q}}=\bunderline{\mathbf{F}}
\end{equation}
Each summation is defined as 
\begin{equation}
\left\{\begin{array}{rlrl}
\bunderline{\mathbf{M}}&=\sum(\bunderline{\mathbf{M}}^e+\bunderline{\mathbf{M}}^b+\bunderline{\mathbf{M}}^d)&\bunderline{\mathbf{D}}&=\sum(\eta_v\bunderline{\mathbf{K}}^e+\bunderline{\mathbf{G}}^e+\bunderline{\mathbf{D}}^b+\bunderline{\mathbf{G}}^d)\\
\bunderline{\mathbf{K}}&=\sum(\eta_a\bunderline{\mathbf{K}}^e+\eta_b\bunderline{\mathbf{C}}^e+\bunderline{\mathbf{K}}^b)&\vec{\mathbf{F}}&=\vec{\mathbf{F}}^d
\end{array}\right.
\end{equation}
Care must be taken here to associate the correct degrees of freedom, and recognize that some of the matrices are elemental and others are nodal. 
\subsection{Code Implementation}
\subsection{Model Reduction}
\subsubsection{Static Condensation}
\section{Model Analysis}
\subsection{Frequency Response}
\subsection{Stability Analysis}
\subsection{Shapes}
\subsection{Campbell}
\subsection{RootLocus}
