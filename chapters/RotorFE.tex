\chapter{Rotor FE}
\section{Beam Element FE Equation}
\subsection{Bernoulli-Euler Beam equation}
Assumptions used to derive the bernoulli-Euler beam equation are (Complete derivation in \cite{craig2006fundamentals}):
\begin{enumerate}
	\item Beam is bending in a plane, in this case in the y-direction, where the x-direction is along the length of the beam.
	\item The  neutral axis undergoes no deformation in the longitudinal direction.
	\item Cross sections remain plane and perpendicular to the nuetral axis.
	\item The material is linear-elastic.
	\item Stresses in the y and z direction are negligible compared to those in the x direction.
	\item Rotatory inertia may be neglected in moment equation.
	\item Mass density is constant at each cross section, so that each mass center is coincident with the centroid of that section.
\end{enumerate}
Using kinematics and assumptions 2 \& 3, the strain in the x direction may be related to the curvature of the beam, $\mu(x,t)$, and the distance from the neutral axis by
\begin{equation} \label{eq:curvature}
\epsilon=-\dfrac{y}{\mu}
\end{equation}
then, with assumption 4 \& 7 the relation from curvature to moment is
\begin{equation} \label{eq:curv_moment}
M(x,t)=\dfrac{EI}{\mu}
\end{equation}
\begin{figure}
	\centering
	\def\svgwidth{300pt}
	\import{figures/}{BeamFBD.pdf_tex}
	\caption{Free body diagram of a beam section in planar bending.}
	\label{fig:BeamFBD}
\end{figure}
where $ E $, Young's modulus, and $ I $, area moment of inertia are constant in cross sections. By using Newton's laws and the free body diagram of a single beam element, see Figure \ref{fig:BeamFBD}, the equations of motion are summarized as:
\begin{equation} \label{eq:Newton}
\sum{F_y}=\Delta m a_y \quad \textrm{\&} \quad \sum{M_G}=0
\end{equation}
Moment equation is represented as moments summarized at the center of mass, $ G $. The right hand side of moment equation of EOM \eqref{eq:Newton} is know to be null due to assumption 6. Applying Newton's equations \eqref{eq:Newton} to the FBD of Figure \ref{fig:BeamFBD} results in the force equation
\begin{equation} \label{eq:force_EOM}
F(x,t)-F(x+\Delta x,t)=\rho A\Delta x \frac{\partial^2 v}{\partial t^2}
\end{equation}
and moment equation
\begin{equation} \label{eq:moment_EOM}
-M(x,t) + M(x+\Delta x,t) + F(x,t)\left( \frac{-\Delta x}{2}\right)  + \left[-F(x+\Delta x,t)\right] \left(\frac{\Delta x}{2}\right) = 0
\end{equation}
Taking the limit of equations \eqref{eq:force_EOM} \& \eqref{eq:moment_EOM} as $ \Delta x \rightarrow 0 $ results in equations \eqref{eq:force_EOM_partial} \& \eqref{eq:moment_EOM_partial} respectively.
\begin{equation} \label{eq:force_EOM_partial}
\frac{\partial F}{\partial x}=-\rho A \frac{\partial^2 v}{\partial t^2}
\end{equation} 
\begin{equation} \label{eq:moment_EOM_partial}
\frac{\partial M}{\partial x} = F
\end{equation}
Assuming the beam slope, $ \frac{\partial v}{\partial x} $, remains relatively small, then linearized curvature of the beam is inversely proportional to $ \frac{\partial^2 v}{\partial x^2} $. Substituting this linearized curvature in \eqref{eq:curv_moment} produces
\begin{equation} \label{eq:moment_curvature_partial}
M(x,t)=EI\frac{\partial^2 v}{\partial x^2}
\end{equation}
Using linearized moment equation \eqref{eq:moment_curvature_partial}, combined with \eqref{eq:force_EOM_partial} \&  \eqref{eq:moment_EOM_partial} lends the Euler beam equation
\begin{equation} \label{eq:euler_beam_equation}
\frac{\partial^2}{\partial x^2} \left(EI\frac{\partial^2 v}{\partial x^2}\right) = -\rho A\frac{\partial^2 v}{\partial t^2}
\end{equation}
\begin{equation} \label{eq:Hooke's_Law}
T=k\delta
\end{equation}
\begin{equation} \label{eq:Potential_energy_gov}
\pi_p=\frac{1}{2}k(u_2 -u_1)^2-f_{1x} u_1-f_{2x}u_2
\end{equation}

This is the governing differential equation for transverse motion of a slender beam. This equation is not suitable for an application involving lengths that are not much greater than the width of the beam \cite{genta2007dynamics}.

\begin{equation} \label{eq:FEGoverning}
u(x,y,z,t)=N(x,y,z)q(t)
\end{equation}

\subsection{Timoshenko Beam Equations}

\begin{figure}
	\centering
	\def\svgwidth{400pt}
	\import{figures/}{kinematicbeamsurface.pdf_tex}
	\caption{Beam Element with nodal displacements.}
	\label{fig:KineBeamElem}
\end{figure}
\begin{equation}\label{eq:r}
\vec{r}=\vec{r}_c+\vec{t}
\end{equation}
\begin{equation}\label{eq:rprime}
\vec{r}^\prime=\vec{r}_c^\prime+\vec{t}^\prime
\end{equation}
\begin{equation}\label{eq:trot}
\vec{t}^\prime=\underline{R}\vec{t}
\end{equation}
\begin{equation}\label{eq:rtrans}
\vec{r}_c^\prime=\vec{r}_c+\vec{u}_c
\end{equation}
\begin{equation}\label{eq:RotTransformation}
\underline{R}=exp\left(\tilde{\underline{\Psi}}\right), \quad where\quad \tilde{\underline{\Psi}}=\left[\begin{array}{ccc}
0&-\theta&\psi\\
\theta&0&-\phi\\
-\psi&\phi&0
\end{array}\right]
\end{equation}
\begin{equation}\label{eq:RotTransformationExpanded}
R=\underline{I}+\tilde{\underline{\Psi}}+1/2\left(\tilde{\underline{\Psi}}\right)^2+...
\end{equation}
\begin{equation}\label{eq:RotTransformationApprox}
R\approx\underline{I}+\tilde{\underline{\Psi}}
\end{equation}
\begin{equation}\label{eq:DispVectorrpr}
\vec{u}=\vec{r^\prime}-\vec{r}
\end{equation}
\begin{equation}\label{eq:DispVectorrptprt}
\vec{u}=\vec{r_c^\prime}+\vec{t^\prime}-\vec{r}_c+\vec{t}
\end{equation}
\begin{equation}\label{eq:DispVectRotTranst}
\vec{u}=\vec{u}_c+(\underline{R}-\underline{I})\vec{t}
\end{equation}
\begin{equation}\label{DispVectExpanded}
\vec{u} = \left\{\begin{array}{c}
	u\\
	v\\
	w\end{array}\right\}=\left\{\begin{array}{c}
	u_c\\
	v_c\\
	w_c\end{array}\right\}+\left[\begin{array}{ccc}
	0&-\theta&\psi\\
	\theta&0&-\phi\\
	-\psi&\phi&0
	\end{array}\right]\left\{\begin{array}{c}
	0\\
	y\\
	z\end{array}\right\}
\end{equation}
\begin{equation}\label{eq:DispVectEvaluated}
\vec{u}=\left\{\begin{array}{c}
u_c-\theta y+\psi z\\
v_c-\phi z\\
w_c+\phi y\end{array}\right\}
\end{equation}
\begin{equation}\label{eq:LinearStrainDispRelationship}
2\epsilon_{ij}=u_{i,j}+u_{j,i}
\end{equation}
\begin{equation}\label{eq:StrainEvaluated}
\underline{\epsilon}=\frac{1}{2}\left[\begin{array}{ccc}
2\left(\frac{\partial u}{\partial x}-\frac{\partial\theta}{\partial x}y+\frac{\partial\psi}{\partial x}z\right) & \frac{\partial v}{\partial x}-\frac{\partial\phi}{\partial x}z-\theta & \frac{\partial w}{\partial x}+\frac{\partial \phi}{\partial x}y+\psi\\
\frac{\partial v}{\partial x}-\frac{\partial\phi}{\partial x}z-\theta &0&0\\
\frac{\partial w}{\partial x}+\frac{\partial \phi}{\partial x}y+\psi&0&0\end{array}\right]
\end{equation}
\begin{equation}\label{LinearElasticStressStrainRelationship}
\sigma_{ij}=\lambda\delta_{ij}\epsilon_{kk}+2\mu\epsilon_{ij}
\end{equation}
\begin{equation}\label{StressEvaluated}
\underline{\sigma}=\left[\begin{array}{ccc}
(\lambda+2\mu)(\frac{\partial u}{\partial x}-\frac{\partial\theta}{\partial x}y+\frac{\partial\psi}{\partial x}z)&\mu(\frac{\partial v}{\partial x}-\frac{\partial\phi}{\partial x}z-\theta)&\mu(\frac{\partial w}{\partial x}+\frac{\partial\phi}{\partial x}y+\psi)\\
\mu(\frac{\partial v}{\partial x}-\frac{\partial\phi}{\partial x}z-\theta)&0&0\\
\mu(\frac{\partial w}{\partial x}+\frac{\partial\phi}{\partial x}y+\psi)&0&0
\end{array}\right]
\end{equation}
%\begin{equation}\label{key}
%\sigma_{ij}=\lambda\delta_{ij}\epsilon_{k,k}+\mu\left(u_{i,j}+u_{j,i}\right)
%\end{equation}
\begin{equation}\label{eq:TotalVirtualStrainEnergyExpression}
\delta V = \int_{\volume}\sigma_{ij}\delta\epsilon_{ij}d\volume
\end{equation}
Now the inner product of the total virtual strain energy can be expanded, and it is realized that each term of the expression in the integral can be set equal to zero.
\begin{equation}\label{eq:TotalVirtualStrainEnergyExpanded}
\left.\begin{array}{rc}
\sigma_{xx}\delta\epsilon_{xx}:&(\lambda+2\mu)(\frac{\partial u}{\partial x}-\frac{\partial\theta}{\partial x}y+\frac{\partial\psi}{\partial x}z)\delta\epsilon_{xx}=0\\
\sigma_{xy}\delta\epsilon_{xy}+\sigma_{yx}\delta\epsilon_{yx}=2\sigma_{xy}\delta\epsilon_{xy}:&\mu(\frac{\partial v}{\partial x}-\frac{\partial\phi}{\partial x}z-\theta)\delta\epsilon_{xy}=0\\
\sigma_{xz}\delta\epsilon_{xz}+\sigma_{zx}\delta\epsilon_{zx}=2\sigma_{xz}\delta\epsilon_{xz}:&\mu(\frac{\partial w}{\partial x}+\frac{\partial\phi}{\partial x}y+\psi)\delta\epsilon_{xz}=0
\end{array}\right\}
\end{equation}
Then expand virtual strains into the generalized strains corresponding to the degrees of freedom

%In the element depicted in Figure \ref{fig:BeamElem} the generalized strains can be identified as 
%\begin{empheq}[left=\empheqlbrace]{align}
%&\kappa_x=-\theta^\prime \label{eq:kx}\\
%&\kappa_y=\phi^\prime \label{eq:ky}\\
%&\gamma_x=u^\prime -\theta \label{eq:gx}\\
%&\gamma_y=v^\prime +\phi \label{eq:gy}
%\end{empheq}
%where $\kappa$ \& $\gamma$ represent the curvature and shear strain respectively. Combining these generalized strains leads to summarize non-trivial strains as
%\begin{empheq}[left=\empheqlbrace]{align}
%&\epsilon_{zz}=\kappa_xx + \kappa_yy \label{eq:ezz}\\
%&\epsilon_{zx}=\gamma_x \label{eq:ezx}\\
%&\epsilon_{zy}=\gamma_y  \label{eq:ezy}
%\end{empheq}
%Using the principle of virtual displacements expresses virtual strain energy as
%\begin{equation} \label{eq:virtual_strain_energy}
%\delta U=\int_{\volume}(\sigma_{zz}\delta\epsilon_{zz}+\sigma_{zx}\delta\epsilon_{zx}+\sigma_{zy}\delta\epsilon_{zy})d\volume
%\end{equation}
%\begin{figure}
%	\centering
%	\def\svgwidth{400pt}
%	\import{figures/}{BeamElement.pdf_tex}
%	\caption{Beam Element with nodal displacements.}
%	\label{fig:BeamElem}
%\end{figure}