\chapter{Rotor FE}
\section{Beam Element FE Equation}
\subsection{Bernoulli-Euler Beam equation}

Assumptions used to derive the bernoulli-Euler beam equation are (Complete derivation in \cite{craig2006fundamentals}):
\begin{enumerate}
	\item Beam is bending in a plane, in this case in the y-direction, where the x-direction is along the length of the beam.
	\item The  neutral axis undergoes no deformation in the longitudinal direction.
	\item Cross sections remain plane and perpendicular to the nuetral axis.
	\item The material is linear-elastic.
	\item Stresses in the y and z direction are negligible compared to those in the x direction.
	\item Rotatory inertia may be neglected in moment equation.
	\item Mass density is constant at each cross section, so that each mass center is coincident with the centroid of that section.
\end{enumerate}
Using kinematics and assumptions 2 \& 3, the strain in the x direction may be related to the curvature of the beam, $\mu(x,t)$, and the distance from the neutral axis by
\begin{equation} \label{eq:curvature}
\epsilon=-\dfrac{y}{\mu}
\end{equation}
then, with assumption 4 \& 7 the relation from curvature to moment is
\begin{equation} \label{eq:curv_moment}
M(x,t)=\dfrac{EI}{\mu}
\end{equation}
\begin{figure}
	\centering
	\def\svgwidth{300pt}
	\import{figures/}{BeamFBD.pdf_tex}
	\caption{Free body diagram of a beam section in planar bending.}
	\label{fig:BeamFBD}
\end{figure}
where $ E $, Young's modulus, and $ I $, area moment of inertia are constant in cross sections. By using Newton's laws and the free body diagram of a single beam element, see Figure \ref{fig:BeamFBD}, the equations of motion are summarized as:
\begin{equation} \label{eq:Newton}
\sum{F_y}=\Delta m a_y \quad \textrm{\&} \quad \sum{M_G}=0
\end{equation}
Moment equation is represented as moments summarized at the center of mass, $ G $. The right hand side of moment equation of EOM \eqref{eq:Newton} is know to be null due to assumption 6. Applying Newton's equations \eqref{eq:Newton} to the FBD of Figure \ref{fig:BeamFBD} results in the force equation
\begin{equation} \label{eq:force_EOM}
F(x,t)-F(x+\Delta x,t)=\rho A\Delta x \frac{\partial^2 v}{\partial t^2}
\end{equation}
and moment equation
\begin{equation} \label{eq:moment_EOM}
-M(x,t) + M(x+\Delta x,t) + F(x,t)\left( \frac{-\Delta x}{2}\right)  + \left[-F(x+\Delta x,t)\right] \left(\frac{\Delta x}{2}\right) = 0
\end{equation}
Taking the limit of equations \eqref{eq:force_EOM} \& \eqref{eq:moment_EOM} as $ \Delta x \rightarrow 0 $ results in equations \eqref{eq:force_EOM_partial} \& \eqref{eq:moment_EOM_partial} respectively.
\begin{equation} \label{eq:force_EOM_partial}
\frac{\partial F}{\partial x}=-\rho A \frac{\partial^2 v}{\partial t^2}
\end{equation} 
\begin{equation} \label{eq:moment_EOM_partial}
\frac{\partial M}{\partial x} = F
\end{equation}
Assuming the beam slope, $ \frac{\partial v}{\partial x} $, remains relatively small, then linearized curvature of the beam is inversely proportional to $ \frac{\partial^2 v}{\partial x^2} $. Substituting this linearized curvature in \eqref{eq:curv_moment} produces
\begin{equation} \label{eq:moment_curvature_partial}
M(x,t)=EI\frac{\partial^2 v}{\partial x^2}
\end{equation}
Using linearized moment equation \eqref{eq:moment_curvature_partial}, combined with \eqref{eq:force_EOM_partial} \&  \eqref{eq:moment_EOM_partial} lends the Euler beam equation
\begin{equation} \label{eq:euler_beam_equation}
\frac{\partial^2}{\partial x^2} \left(EI\frac{\partial^2 v}{\partial x^2}\right) = -\rho A\frac{\partial^2 v}{\partial t^2}
\end{equation}
\begin{equation} \label{eq:Hooke's_Law}
T=k\delta
\end{equation}
\begin{equation} \label{eq:Potential_energy_gov}
\pi_p=\frac{1}{2}k(u_2 -u_1)^2-f_{1x} u_1-f_{2x}u_2
\end{equation}

This is the governing differential equation for transverse motion of a slender beam. This equation is not suitable for an application involving lengths that are not much greater than the width of the beam \cite{genta2007dynamics}.


\subsection{Timoshenko Beam Equations}
\begin{figure}
	\centering
	\def\svgwidth{300pt}
	\import{figures/}{TimoBeamDOF.pdf_tex}
	\caption{Timoshenko beam section with degrees of freedom at some point x along beam axis.}
	\label{fig:TimoBeamDOF}
\end{figure}

\begin{figure}
	\centering
	\def\svgwidth{400pt}
	\import{figures/}{kinematicbeamsurface.pdf_tex}
	\caption{Beam Element with nodal displacements.}
	\label{fig:KineBeamElem}
\end{figure}
\begin{equation}\label{eq:r}
\vec{r}=\vec{r}_c+\vec{t}
\end{equation}
\begin{equation}\label{eq:rprime}
\vec{r}^\prime=\vec{r}_c^\prime+\vec{t}^\prime
\end{equation}
\begin{equation}\label{eq:trot}
\vec{t}^\prime=\underline{R}\vec{t}
\end{equation}
\begin{equation}\label{eq:rtrans}
\vec{r}_c^\prime=\vec{r}_c+\vec{u}_c
\end{equation}
The linearized first order rotational matrix for small angles is represented by $\underline{R}$
\begin{equation}\label{eq:RotTransformation}
\underline{R}=\left[\begin{array}{ccc}
1&-\theta&\psi\\
\theta&1&-\phi\\
-\psi&\phi&1
\end{array}\right]
\end{equation}
%\begin{equation}\label{eq:RotTransformationExpanded}
%R=\underline{I}+\tilde{\underline{\Psi}}+1/2\left(\tilde{\underline{\Psi}}\right)^2+...
%\end{equation}
%\begin{equation}\label{eq:RotTransformationApprox}
%R\approx\underline{I}+\tilde{\underline{\Psi}}
%\end{equation}
\begin{equation}\label{eq:DispVectorrpr}
\vec{u}=\vec{r^\prime}-\vec{r}
\end{equation}
\begin{equation}\label{eq:DispVectorrptprt}
\vec{u}=\vec{r_c^\prime}+\vec{t^\prime}-\vec{r}_c+\vec{t}
\end{equation}
\begin{equation}\label{eq:DispVectRotTranst}
\vec{u}=\vec{u}_c+(\underline{R}-\underline{I})\vec{t}
\end{equation}
\begin{equation}\label{DispVectExpanded}
\vec{u} = \left\{\begin{array}{c}
	u\\
	v\\
	w\end{array}\right\}=\left\{\begin{array}{c}
	u_c\\
	v_c\\
	w_c\end{array}\right\}+\left[\begin{array}{ccc}
	0&-\theta&\psi\\
	\theta&0&-\phi\\
	-\psi&\phi&0
	\end{array}\right]\left\{\begin{array}{c}
	0\\
	y\\
	z\end{array}\right\}
\end{equation}
Therefore, the motion of any point on the beam may be approximated with
\begin{equation}\label{eq:DispVectEvaluated}
\vec{u}=\left\{\begin{array}{c}
u_c-\theta y+\psi z\\
v_c-\phi z\\
w_c+\phi y\end{array}\right\}
\end{equation}
\subsection{Constitutive Relation}
Stresses are assumed to exist in the beam in the axial direction, and in shear on the face of the beam section. Stresses in the transverse, or the y and z, directions are assumed negligible. Shear stresses out of the plane section are assumed to be vanishing as the differential element shrinks. Written out, these stresses are represented by the matrix
\begin{equation}\label{key}
\sigma_{ij}=\left[\begin{array}{ccc}
\sigma_{xx}&\sigma_{xy}&\sigma_{xz}\\
\sigma_{xy}&0&0\\
\sigma_{xz}&0&0
\end{array}\right]
\end{equation}
using the Hooke's Law for a linear elastic isotropic material, expressed as
\begin{equation}\label{key}
\epsilon_{ij}=\frac{1}{E}[(1+\nu)\sigma_{ij}-\nu\delta_{ij}\sigma_{kk}]
\end{equation}
allows the expression of the stress strain relationship in engineering notation as
\begin{equation}\label{eq:ConstitutiveRelationshipEngineeringNotation}
\left\{\begin{array}{c}
\sigma_{xx}\\ \sigma_{xy}\\ \sigma_{xz}
\end{array}\right\} = \left[\begin{array}{ccc}
E&0&0\\0&2G&0\\0&0&2G
\end{array}\right] \left\{\begin{array}{c}
\epsilon_{xx}\\ \epsilon_{xy}\\ \epsilon_{xz}
\end{array}\right\}
\end{equation}
where, $ G=\frac{E}{2(1+\nu)} $. Strains are derived from displacements of equation \ref{eq:DispVectEvaluated} using a linear strain-displacement relationship $ 2\epsilon_{ij}=u_{i,j}+u_{j,i} $.
%\begin{equation}\label{eq:LinearStrainDispRelationship}
%
%\end{equation}
%\begin{equation}\label{eq:StrainEvaluated}
%\underline{\epsilon}=\frac{1}{2}\left[\begin{array}{ccc}
%2\left(\frac{\partial u_c}{\partial x}-\frac{\partial\theta}{\partial x}y+\frac{\partial\psi}{\partial x}z\right) & \frac{\partial v_c}{\partial x}-\frac{\partial\phi}{\partial x}z-\theta & \frac{\partial w_c}{\partial x}+\frac{\partial \phi}{\partial x}y+\psi\\
%\frac{\partial v_c}{\partial x}-\frac{\partial\phi}{\partial x}z-\theta &0&0\\
%\frac{\partial w_c}{\partial x}+\frac{\partial \phi}{\partial x}y+\psi&0&0\end{array}\right]
%\end{equation}
The subscript c is dropped and $ u $,$ v $, and $ w $ are taken to be representative of the displacements on the beam axis. Since all strains will be collapsed into strain energy of the entire beam cross section. 
\begin{equation}\label{eq:StrainEvaluatedSimple}
\left\{\arraycolsep=1pt\begin{array}{rl}
\epsilon_{xx}&=u'-\theta'y+\psi'z\\
\epsilon_{xy}&=\frac{1}{2}(v'-\phi'z-\theta)\\
\epsilon_{xz}&=\frac{1}{2}(w'+\phi'y+\psi)
\end{array}\right.
\end{equation}
Note that in the case of the Euler-Bernoulli beam derivation the slope in the y direction displacement is equal to the face angle about the z-axis, i.e., $v'=\theta$ and $w'=\psi$.\

It will be proven useful to introduce generalized strains that group strain contributions above as axial, bending, torsion, and shear, represented by the symbols $ \varepsilon $, $ \rho $, $ \varphi $, $ \gamma $,respectively.
\begin{equation}\label{eq:GeneralizedStrains}
\left\{\arraycolsep=1pt\begin{array}{rl}
\varepsilon&=u'\\
\rho_{y}&=-\theta'\\
\rho_{z}&=\psi'\\
\varphi&=\phi'\\
\gamma_{y}&=v'-\theta\\
\gamma_{z}&=w'+\psi\end{array}\right.
\end{equation}
allowing the representation of the strains from eq \ref{eq:StrainEvaluatedSimple} using the generalized strains as:
\begin{equation}\label{eq:StrainEvaluatedSimpleGeneralized}
\left\{\begin{array}{l}
\epsilon_{xx}=\varepsilon+\rho_yy+\rho_zz\\
\epsilon_{xy}=\frac{1}{2}(\gamma_y-\varphi z)\\
\epsilon_{xz}=\frac{1}{2}(\gamma_z+\varphi y)
\end{array}\right.
\end{equation}
Now the stresses in equation \ref{eq:ConstitutiveRelationshipEngineeringNotation} can be represented with the displacements as 
\begin{equation}\label{key}
\left\{\begin{array}{c}
\sigma_{xx}\\ \sigma_{xy}\\ \sigma_{xz}
\end{array}\right\} = \left\{\begin{array}{c}
E(u'-\theta'y+\psi'z)\\ G(v'-\phi'z-\theta)\\ G(w'+\phi'y+\psi)
\end{array}\right\} = \left\{\begin{array}{c}
E(\varepsilon+\rho_yy+\rho_zz)\\ G(\gamma_y-\varphi z)\\ G(\gamma_z+\varphi y)
\end{array}\right\}
\end{equation}
%Stress at the beam axis is expressed using Hooke's Law for linear isotropic homogeneous material
%\begin{equation}\label{LinearElasticStressStrainRelationship}
%\sigma_{ij}=\lambda\delta_{ij}\epsilon_{kk}+2\mu\epsilon_{ij}
%\end{equation}
%expanded out 
%\begin{equation}\label{StressEvaluated}
%\underline{\sigma}=\left[\begin{array}{ccc}
%(\lambda+2\mu)(u'-\theta'y+\psi'z)&\mu(v'-\phi'z-\theta)&\mu(w'+\phi'y+\psi)\\
%\mu(v'-\phi'z-\theta)&\lambda(u'-\theta'y+\psi'z)&0\\
%\mu(w'+\phi'y+\psi)&0&\lambda(u'-\theta'y+\psi'z)
%\end{array}\right]
%\end{equation}
%\begin{equation}\label{key}
%\sigma_{ij}=\lambda\delta_{ij}\epsilon_{k,k}+\mu\left(u_{i,j}+u_{j,i}\right)
%\end{equation}
\begin{equation}\label{eq:TotalVirtualStrainEnergyExpression}
\delta U = \int_{\volume}\sigma_{ij}\delta\epsilon_{ij}d\volume
\end{equation}
Now the inner product of the total virtual strain energy can be expanded
\begin{equation}\label{eq:TotalStrainEnergyExpanded}
U=\int_{\volume}\left[\sigma_{xx}\epsilon_{xx}+2\sigma_{xy}\epsilon_{xy}+2\sigma_{xz}\epsilon_{xz}\right]d\volume
\end{equation}
Then expand virtual strains into the generalized strains corresponding to the degrees of freedom of the beam element and collect terms on generalized strains
\begin{equation}\label{eq:TotalStrainEnergyGeneralized}
 U=\int_{\volume}\left[\sigma_{xx}\varepsilon+\sigma_{xx}y\rho_y+\sigma_{xx}z\rho_z+\sigma_{xy}\gamma_y+\sigma_{xz}\gamma_z+(\sigma_{xz}y-\sigma_{xy}z)\varphi\right]d\volume
\end{equation}
recognize stresses conjugate with each generalized strain as the corresponding stress for that phenomena, and integrating to determine the forces and moments related to the generalized strains.
\begin{equation}\label{eq:generalizedForces}
\left\{\arraycolsep=1pt\begin{array}{rcl}
N=&\int_A\sigma_{xx}dA&=E( A\varepsilon+S_y\rho_y+S_z\rho_z)\\
M_y=&\int_A\sigma_{xx}zdA&=E(S_z\varepsilon+I_{xy}\rho_y+I_y\rho_z)\\
M_z=&\int_A\sigma_{xx}ydA&=E(S_y\varepsilon+I_z\rho_y+I_{xy}\rho_z)\\
Q_y=&\int_A\sigma_{xy}dA&=\kappa G(A\gamma_y-S_z\varphi)\\
Q_z=&\int_A\sigma_{xz}dA&=\kappa G(A\gamma_z+S_y\varphi)\\
M_x=&\int_A(\sigma_{xz}y-\sigma_{xy}z)dA&=\kappa G\left(A_y\gamma_z-A_z\gamma_y+J_x\varphi\right)
\end{array}\right.
\end{equation}
\begin{center}where, $ \left\{\begin{array}{l}
\ \kappa=\frac{6(1+\nu)}{7+6\nu}, \textit{for circular cross sections.}\\
\arraycolsep=.75pt\begin{array}{rlccccrl}
A&=\int_AdA&&&&&S_y&=\int_A ydA\\
S_z&=\int_AzdA&&&&&I_y&=\int_Az^2dA\\
I_z&=\int_Ay^2dA&&&&&J_x&=I_y+I_z
\end{array}
\end{array}\right. $\\\end{center}
$ \kappa $ is the shear coefficient which attempts to correct for the fact that the shear strain is not constant over the beam cross section. Assuming the central axis of the beam is coincident with the shear center, then $ A_y=A_z=I_{xy}=0 $. Which simplifies the conjugate forces to


\begin{equation}\label{eq:generalizedForcesSimplified}
\left\{\arraycolsep=1pt\begin{array}{rll}
N&=EA\varepsilon&=EAu'\\
M_y&=EI_y\rho_z&=EI_y\psi'\\
M_z&=EI_z\rho_y&=-EI_z\theta'\\
Q_y&=\kappa G A\gamma_y&=\kappa G A(v'-\theta)\\
Q_z&=\kappa G A\gamma_z&=\kappa G A(w'+\psi)\\
M_x&=\kappa G J_x\varphi&=\kappa G J_x\phi'
\end{array}\right.
\end{equation}

%\begin{equation}\label{eq:TotalVirtualStrainEnergyGeneralizedExpanded}
%\delta U=\int_{0}^{l}\left[N\delta\varepsilon+M_z\delta\rho_y+M_y\delta\rho_z+Q_y\delta\gamma_y+Q_z\delta\gamma_z+M_x\delta\varphi\right]dx
%\end{equation}
%upon expanding with forces defined in eq \ref{eq:generalizedForcesSimplified}
%\begin{multline}
%\delta U=\int_{0}^{l}[EAu'\delta u'
%+EI_y\psi'\delta\psi'
%+EI_z\theta'\delta\theta'
%+\kappa G A(v'-\theta)(\delta v'-\delta\theta)\\
%+\kappa G A(w'+\psi)(\delta w'+\delta\psi)
%+\kappa G J_x\phi'\delta\phi']dx
%\end{multline}
%and using integration by parts to achieve a relation to the displacements 
%\begin{multline}\label{eq:TotalVirtualStrainEnergyDisplacementRel}
%\delta U=\int_{0}^{l}\{EAu''\delta u
%+\kappa G A(v''-\theta')\delta v
%+\kappa G A(w''+\psi')\delta w\\
%+[EI_y\psi''-\kappa G A(w'+\psi)]\delta\psi
%+[EI_z\theta''+\kappa G A(v'-\theta)]\delta\theta\\
%+\kappa G J_x\phi''\delta\phi\}dx + \delta U_b
%\end{multline}
%leaving $ \delta U_b $ as the boundary value terms to be neglected. The principle of virtual work for a quasistatic beam with no external work gives $ \delta U=0 $. Therefore, each term in the integrand of \ref{eq:TotalVirtualStrainEnergyDisplacementRel} must equal zero.
\subsection{Equations of Motion}
Now the equations are motion are derived for the Timoshenko beam element. The derivation is the same as for a Euler-Bernoulli beam with the exception of the constitutive relations combined at the end.
\begin{figure}
	\centering
	\def\svgwidth{600pt}
	\import{figures/}{BeamDifferentialSection.pdf_tex}
	\caption{Beam differential element with generalized forces.}
	\label{fig:BeamDifferentialSection}
\end{figure}
Using conservation of momentum and conservation of the moment of momentum a relationship between inertia and internal forces is developed. Applying summation of forces in the x-direction
\begin{equation}\label{key}
(N+\frac{1}{2}\frac{\partial N}{\partial x}dx)-(N-\frac{1}{2}\frac{\partial N}{\partial x}dx)=\rho A dx \frac{\partial^2u}{\partial x^2}
\end{equation}
which simplifies to
\begin{equation}\label{key}
\frac{\partial N}{\partial x}=\rho A\frac{\partial^2 u}{\partial x^2}
\end{equation}
similarly for the other directions, and summing moments about the center of mass
\begin{equation}\label{eq:EquilibriumEquations}
\left\{\arraycolsep=1pt\begin{array}{rl}
\frac{\partial Q_y}{\partial x}&=\rho A\frac{\partial^2 v}{\partial x^2}\\
\frac{\partial Q_z}{\partial x}&=\rho A\frac{\partial^2 w}{\partial x^2}\\
\frac{\partial M_y}{\partial x}-Q_z&=\rho I_y \frac{\partial^2\psi}{\partial x^2} + \rho J_x\Omega\frac{\partial\theta}{\partial x}\\
\frac{\partial M_z}{\partial x}+Q_y&=\rho I_z \frac{\partial^2\theta}{\partial x^2} - \rho J_x\Omega\frac{\partial\psi}{\partial x}\\
\frac{\partial M_x}{\partial x}&=\rho J_x\frac{\partial^2\phi}{\partial x^2}
\end{array}\right.
\end{equation}
The generalized forces of equation \ref{eq:generalizedForcesSimplified} are substituted in the equilibrium equations \ref{eq:EquilibriumEquations}
%\begin{equation}\label{eq:EquationsOfMotionIndependent}
%\left\{\arraycolsep=1pt\begin{array}{rl}
%EAu''&=\rho A\ddot{u}\\
%\kappa GA(v''-\theta')&=\rho A\ddot{v}\\
%\kappa GA(w''+\psi')&=\rho A\ddot{w}\\
%EI_y\psi''-\kappa GA(w'-\theta)&=\rho I_y\ddot{\psi}+\rho J_x\Omega\dot{\theta}\\
%EI_z\theta''+\kappa GA(v'+\psi)&=\rho I_z\ddot{\theta}-\rho J_x\Omega\dot{\psi}\\
%\kappa GJ_z\phi''&=\rho J_x\ddot{\phi}
%\end{array}\right.
%\end{equation}
\begin{subequations}\label{eq:EquationsOfMotionIndependent}
\begin{empheq}[left={\empheqlbrace\,}]{align}
EAu''&=\rho A\ddot{u}\label{eq:EquationsOfMotionIndependent_u}\\
\kappa GA(v''-\theta')&=\rho A\ddot{v}\label{eq:EquationsOfMotionIndependent_v}\\
\kappa GA(w''+\psi')&=\rho A\ddot{w}\label{eq:EquationsOfMotionIndependent_w}\\
EI_y\psi''-\kappa GA(w'-\theta)&=\rho I_y\ddot{\psi}+\rho J_x\Omega\dot{\theta}\label{eq:EquationsOfMotionIndependent_psi}\\
EI_z\theta''+\kappa GA(v'+\psi)&=\rho I_z\ddot{\theta}-\rho J_x\Omega\dot{\psi}\label{eq:EquationsOfMotionIndependent_theta}\\
\kappa GJ_z\phi''&=\rho J_x\ddot{\phi}\label{eq:EquationsOfMotionIndependent_phi}
\end{empheq}
\end{subequations}
In Matrix form, this system of equations can be represented by this equation
\begin{equation}\label{key}
\bunderline{\mathfrak{M}}\ddot{\vec{\mathbf{u}}}+\bunderline{\mathfrak{G}}\dot{\vec{\mathbf{u}}}-\frac{\partial}{\partial x}\bunderline{\mathcal{S}}\vec{\mathbf{u}}-\bunderline{\mathcal{P}}\bunderline{\mathcal{S}}\vec{\mathbf{u}}
\end{equation}
where,
\begin{equation}\label{key}
\def\cs{2em}
\newcommand{\SetToWidest}[1]{\makebox[\cs]{$#1$}}%
\left\{\def\arraystretch{1}\begin{array}{@{}rlrl}
\bunderline{\mathfrak{M}}&=\left[\arraycolsep=-.5pt\begin{array}{cccccc}
\SetToWidest{\rho A}&\SetToWidest{0}&\SetToWidest{0}&\SetToWidest{0}&\SetToWidest{0}&\SetToWidest{0}\\
0&\rho A&0&0&0&0\\
0&0&\rho I_y&0&0&0\\
0&0&0&\rho I_z&0&0\\
0&0&0&0&\rho A&0\\
0&0&0&0&0&\rho J_x
\end{array}\right]&\bunderline{\mathfrak{G}}&=\left[\arraycolsep=-.5pt\begin{array}{cccccc}
\makebox[\cs]{0}&\makebox[\cs]{0}&\makebox[\cs]{0}&\makebox[\cs]{0}&\makebox[\cs]{0}&\makebox[\cs]{0}\\
0&0&0&0&0&0\\
0&0&0&\rho J_x\Omega&0&0\\
0&0&\rho J_x\Omega&0&0&0\\
0&0&0&0&0&0\\
0&0&0&0&0&0
\end{array}\right]\\
&&\\[-1em]
\bunderline{\mathcal{P}}&=\left[\arraycolsep=-.5pt\begin{array}{cccccc}
\SetToWidest{0}&\SetToWidest{0}&\SetToWidest{0}&\SetToWidest{0}&\SetToWidest{0}&\SetToWidest{0}\\
0&0&0&0&0&0\\
0&-1&0&0&0&0\\
1&0&0&0&0&0\\
0&0&0&0&0&0\\
0&0&0&0&0&0
\end{array}\right]&\bunderline{\mathcal{S}}&=\left[\arraycolsep=1.8pt\begin{array}{cccccc}
\makebox[\cs]{$\kappa GA \frac{\partial}{\partial x}$}&\makebox[\cs]{0}&\makebox[\cs]{0}&\makebox[\cs]{$-\kappa G \frac{\partial}{\partial x}$}&\makebox[\cs]{0}&\makebox[\cs]{0}\\
0&\kappa GA\frac{\partial}{\partial x}&\kappa GA&0&0&0\\
0&0&EI_y\frac{\partial}{\partial x}&0&0&0\\
0&0&0&EI_z\frac{\partial}{\partial x}&0&0\\
0&0&0&0&EA\frac{\partial}{\partial x}&0\\
0&0&0&0&0&\kappa GJ_x\frac{\partial}{\partial x}
\end{array}\right]
\end{array}\right.
\end{equation}
and $ \vec{\mathbf{u}}=[v,w,\psi,\theta,u,\phi]^{\T} $The principle of virtual displacements is utilized on the equations of motion to and integrated over the length of the beam.
\begin{equation}\label{key}
\int_0^l \delta\vec{\mathbf{u}}^\T\bunderline{\mathfrak{M}}\ddot{\vec{\mathbf{u}}}dx+\int_0^l \delta\vec{\mathbf{u}}^\T\bunderline{\mathfrak{G}}\dot{\vec{\mathbf{u}}}-\int_0^l\delta\vec{\mathbf{u}}^\T\frac{\partial}{\partial x}\bunderline{\mathcal{S}}\vec{\mathbf{u}}dx-\int_0^l\delta\vec{\mathbf{u}}^\T\bunderline{\mathcal{P}}\bunderline{\mathcal{S}}\vec{\mathbf{u}}dx=0
\end{equation}
integration by parts on the third term and replacing $ \underline{\mathcal{S}} $ with $ \bunderline{\mathcal{D}}\bunderline{\mathcal{B}} $, and making use of the Identity matrix, $ \bunderline[2]{\mathbf{I}} $
\begin{equation}\label{key}
\int_0^l \delta\vec{\mathbf{u}}^\T\bunderline{\mathfrak{M}}\ddot{\vec{\mathbf{u}}}dx+\int_0^l \delta\vec{\mathbf{u}}^\T\bunderline{\mathfrak{G}}\dot{\vec{\mathbf{u}}}+\int_0^l\delta\vec{\mathbf{u}}^\T(\frac{\partial}{\partial x}\bunderline[2]{\mathbf{I}}-\bunderline{\mathcal{P}})\bunderline{\mathcal{D}}\bunderline{\mathcal{B}}\vec{\mathbf{u}}dx=0
\end{equation}
\begin{equation}\label{key}
\def\cs{1.5em}
\bunderline{\mathcal{D}}=\left[\arraycolsep=1.8pt\begin{array}{cccccc}
\makebox[\cs]{$\kappa GA$}&\makebox[\cs]{0}&\makebox[\cs]{0}&\makebox[\cs]{0}&\makebox[\cs]{0}&\makebox[\cs]{0}\\
0&\kappa GA&0&0&0&0\\
0&0&EI_y&0&0&0\\
0&0&0&EI_z&0&0\\
0&0&0&0&EA&0\\
0&0&0&0&0&\kappa GJ_x
\end{array}\right] \&\quad \bunderline{\mathcal{B}}=\left[\arraycolsep=1.8pt\begin{array}{cccccc}
\makebox[\cs]{$\frac{\partial}{\partial x}$}&\makebox[\cs]{0}&\makebox[\cs]{0}&\makebox[\cs]{-1}&\makebox[\cs]{0}&\makebox[\cs]{0}\\
0&\frac{\partial}{\partial x}&1&0&0&0\\
0&0&\frac{\partial}{\partial x}&0&0&0\\
0&0&0&\frac{\partial}{\partial x}&0&0\\
0&0&0&0&\frac{\partial}{\partial x}&0\\
0&0&0&0&0&\frac{\partial}{\partial x}
\end{array}\right]
\end{equation}
Notice that $ \frac{\partial}{\partial x}\bunderline{I}-\bunderline{\mathcal{P}}=\bunderline{\mathcal{B}}^\T $ so the equation of motion becomes
\begin{equation}\label{eq:EquationOfMotionVirtual}
\int_0^l \delta\vec{\mathbf{u}}^\T\bunderline{\mathfrak{M}}\ddot{\vec{\mathbf{u}}}dx+\int_0^l \delta\vec{\mathbf{u}}^\T\bunderline{\mathfrak{G}}\dot{\vec{\mathbf{u}}}+\int_0^l\delta\vec{\mathbf{u}}^\T\bunderline{\mathcal{B}}^\T\bunderline{\mathcal{D}}\bunderline{\mathcal{B}}\vec{\mathbf{u}}dx=0
\end{equation}
Now the solution of this differential system motivates a separation of variables that will be discussed in the next section.
\subsection{Shape Functions}
The displacements thus far have been assumed to be functions of both position and time. Now the total displacement is seperated into functions that depend on time and functions that depend on position. This is a fundamental part of the discretization of the beam element, and the use the finite element method. 
\begin{equation} \label{eq:FEGoverning}
\left\{\begin{array}{rl}
\vec{\mathbf{u}}(x,t)&=\bunderline{\mathbf{N}}(x)\vec{\mathbf{q}}(t)\\
\dot{\vec{\mathbf{u}}}(x,t)&=\bunderline{\mathbf{N}}(x)\dot{\vec{\mathbf{q}}}(t)\\
\ddot{\vec{\mathbf{u}}}(x,t)&=\bunderline{\mathbf{N}}(x)\ddot{\vec{\mathbf{q}}}(t)\\
\delta\vec{\mathbf{u}}(x,t)&=\bunderline{\mathbf{N}}(x)\delta\vec{\mathbf{q}}(t)
\end{array}\right.
\end{equation}
where, $ \vec{\mathbf{q}}=[v_1,w_1,\psi_1,\theta_1,v_2,w_2,\psi_2,\theta_2,u_1,\phi_1,u_2,\phi_2]^\T $. The specfic order of $ \vec{\mathbf{q}} $ is chosen with $ u $ and $ \phi $ at the end to ease the condensation of the axial and torsional degrees of freedom. The shape functions $ \bunderline{\mathbf{N}}(x) $ must solve the static portion of the differential equations \ref{eq:EquationsOfMotionIndependent}.
%\begin{subequations}\label{eq:EulerLagrangianEquations}
%	\begin{empheq}[left={\empheqlbrace\,}]{align}
%	&EAu''=0 \label{eq:EulerLagrangianAxial}\\
%	&\kappa GA(v''-\theta')=0 \label{eq:EulerLagrangianSheary}\\
%	&\kappa GA(w''+\psi')=0 \label{eq:EulerLagrangianShearz}\\
%	&\tilde{E}I_y\psi''-\kappa G A(w'+\psi)=0 \label{eq:EulerLagrangianBendingy}\\
%	&\tilde{E}I_z\theta''+\kappa G A(v'-\theta)=0 \label{eq:EulerLagrangianBendingz}\\
%	&\kappa G J_x\phi''=0 \label{eq:EulerLagrangianTorsion}
%	\end{empheq}
%\end{subequations}
%These are the governing differential equations for a Timoshenko beam element.

\begin{figure}
	\centering
	\def\svgwidth{400pt}
	\import{figures/}{BeamElement.pdf_tex}
	\caption{Beam Element with nodal displacements.}
	\label{fig:BeamElem}
\end{figure}
These shape functions are chosen as a polynomials that satisfy the boundary nodal displacements and rotations at the ends of a beam element. These nodal degrees of freedom, depicted in Figure \ref{fig:BeamElem} are considered to be interpolated through the beam element by the shape functions. Interpolation functions chosen are listed in Equation \ref{eq:DisplacementInterpolationFunctionsChosen}. Axial displacement, $ u $, and torsional rotation, $ \phi $ are independent, so their shape functions are chosen as polynomials that satisfy the differential equation. Conversely, transverse displacements, $ v $ \& $ w $, and bending rotations, $ \psi $ \& $ \theta $ are coupled. Polynomial functions are chosen for $ v $ \& $ w $ and their rotational counterparts are derived using the differential relations.
\begin{subequations}\label{eq:DisplacementInterpolationFunctionsChosen}
\begin{empheq}[left={\empheqlbrace\,}]{align}
u&=c_1+c_2x \label{eq:AxialInterpolationFunction}\\
v&=c_3+c_4x+c_5x^2+c_6x^3 \label{eq:TransverseyInterpolationFunction}\\ 
w&=c_7+c_8x+c_9x^2+c_{10}x^3 \label{eq:TransversezInterpolationFunction}\\
\phi&=c_{11}+c_{12}x \label{eq:TorsionInterpolationFunction}
\end{empheq}
\end{subequations}
Using polynomial definition of transverse displacement of equations \ref{eq:TransverseyInterpolationFunction} \& \ref{eq:TransversezInterpolationFunction} in the differential equations \ref{eq:EquationsOfMotionIndependent_v}, \ref{eq:EquationsOfMotionIndependent_w}, \ref{eq:EquationsOfMotionIndependent_psi}, \ref{eq:EquationsOfMotionIndependent_theta} the interpolation functions of bending rotations are derived as:
\begin{subequations}\label{eq:DisplacmentInterpolationFunctionsDerived}
\begin{empheq}[left={\empheqlbrace\,}]{align}
\psi&=K_yc_{10}-c_8-2c_9x-3c_{10}x^2 \label{eq:RotationyInterpolationFunction}\\
\theta&=K_zc_6+c_4+2c_5x+3c_{6}x^2 \label{eq:RotationzInterpolationFunction}
\end{empheq}
\end{subequations}
where, $ K_y=\frac{6EI_y}{\kappa GA} $\& $ K_z=\frac{6EI_z}{\kappa GA} $
\subsection{Boundary Conditions}
Boundary conditions for the interpolation polynomials of equations \ref{eq:DisplacementInterpolationFunctionsChosen} \& \ref{eq:DisplacmentInterpolationFunctionsDerived} are defined as the components of the vector $ \vec{\mathbf{q}} $$ u_j=u(x_j) $ and similarily for other degrees of freedom. Where, $ j=1,2 $ and defines the two states. In this derivation, $ x_1=0 $ and $ x_2=l $. Application of these boundary condition results in this relation between the polynomial constants and the boundary conditions.
\begin{equation}\label{DisplacementInterpolationBoundaryMatrix}
\def\cs{4em}
\newcommand{\WidestEntry}{$\scriptstyle K_y\!-\!3l^2$}%
%\newcommand{\SetToWidest}[1]{\makebox[\widthof{\WidestEntry}]{$#1$}}%
\newcommand{\SetToWidest}[1]{\makebox[\cs]{$#1$}}%
\left\{\def\arraystretch{1}\begin{array}{@{}c@{}}
u_1\\u_2\\v_1\\v_2\\w_1\\w_2\\\psi_1\\\psi_2\\\theta_1\\\theta_2\\\phi_1\\\phi_2
\end{array}\right\}\hspace{-4pt}=\hspace{-4pt}\left[\arraycolsep=-.68em\def\arraystretch{1}\begin{array}{@{}lccccccccccr@{}}
\makebox[\cs/2][l]{1}& \SetToWidest{0}& \SetToWidest{0}& \SetToWidest{0}& \SetToWidest{0}& \SetToWidest{0}& \SetToWidest{0}& \SetToWidest{0}& \SetToWidest{0}& \SetToWidest{0}& \SetToWidest{0}& \makebox[\cs/2][r]{0}\\
1& l& 0& 0& 0& 0& 0& 0& 0& 0& 0& 0\\
0& 0& 1& 0& 0& 0& 0& 0& 0& 0& 0& 0\\
0& 0& 1& l& l^2& l^3& 0& 0& 0& 0& 0& 0\\
0& 0& 0& 0& 0& 0& 1& 0& 0& 0& 0& 0\\
0& 0& 0& 0& 0& 0& 1& l& l^2& l^3& 0& 0\\
0& 0& 0& 0& 0& 0& 0& \text{-}1& 0& \text{-}K_y& 0& 0\\
0& 0& 0& 0& 0& 0& 0& \text{-}1& \text{-}2l& \text{-}K_y\text{-}3l^2& 0& 0\\
0& 0& 0& 1& 0&  K_z& 0& 0& 0& 0& 0& 0\\
0& 0& 0& 1& 2l&  K_z\!\text{+}3l^2& 0& 0& 0& 0& 0& 0\\
0& 0& 0& 0& 0& 0& 0& 0& 0& 0& 1& 0\\
0& 0& 0& 0& 0& 0& 0& 0& 0& 0& 1& l
\end{array}\right]\hspace{-6pt}\left\{\def\arraystretch{1}\begin{array}{@{}c@{}}
c_1\\c_2\\c_3\\c_4\\c_5\\c_6\\c_7\\c_8\\c_9\\c_{10}\\c_{11}\\c_{12}
\end{array}\right\}
\end{equation}
Inversion of this matrix results in a system of equations defining the constant $ c_1 $ through $ c_{12} $. These constants are then substituted in to the polynomial expressions \ref{eq:DisplacementInterpolationFunctionsChosen} \& \ref{eq:DisplacmentInterpolationFunctionsDerived} giving the interpolations as functions of the nodal displacements.
\begin{equation}\label{eq:InterpolationFunctions}
\left\{\begin{array}{l}
u=N_1u_1+N_2u_2\\
v=T_{v_1}v_1+T_{v_2}v_2+T_{\theta_1}\theta_1+T_{\theta_2}\theta_2\\
w=T_{w_1}w_1+T_{w_2}w_2+T_{\psi_1}\psi_1+T_{\psi_2}\psi_2\\
\psi=R_{w_1}w_1+R_{w_2}w_2+R_{\psi_1}\psi_1+R_{\psi_2}\psi_2\\
\theta=R_{v_1}v_1+R_{v_2}v_2+R_{\theta_1}\theta_1+R_{\theta_2}\theta_2\\
\phi=N_1\phi_1+N_2\phi_2
\end{array}\right.
\end{equation}
with;
\begin{equation*}
\left\{\arraycolsep=.2em\begin{array}{@{}ll}
N_1=1-\zeta&N_2=\zeta\\
T_{v_1}=\frac{1}{1+\alpha_z}(2\zeta^3-3\zeta^2-\alpha_z\zeta+1+\alpha_z)&T_{v_2}=\frac{1}{1+\alpha_z}(-2\zeta^3+3\zeta^2+\alpha_z\zeta)\\
T_{\theta_1}=\frac{l}{1+\alpha_z}[\zeta^3-(2+\frac{1}{2}\alpha_z)\zeta^2+(1+\frac{1}{2}\alpha_z)\zeta]&T_{\theta_2}=\frac{l}{1+\alpha_z}[\zeta^3-(1-\frac{1}{2}\alpha_z)\zeta^2-\frac{1}{2}\alpha_z\zeta]\\
T_{w_1}=\frac{1}{1+\alpha_y}(2\zeta^3-3\zeta^2-\alpha_y\zeta+1+\alpha_y)&T_{w_2}=\frac{1}{1+\alpha_y}(-2\zeta^3+3\zeta^2+\alpha_y\zeta)\\
T_{\psi_1}=\frac{l}{1+\alpha_y}[-\zeta^3+(2+\frac{1}{2}\alpha_y)\zeta^2-(1+\frac{1}{2}\alpha_y)\zeta]&T_{\psi_2}=\frac{l}{1+\alpha_y}(-\zeta^3+(1-\frac{1}{2}\alpha_y)\zeta^2+\frac{1}{2}\alpha_y\zeta)\\
R_{w_1}=\frac{6/l}{1+\alpha_y}(-\zeta^2+\zeta)&R_{w_2}=\frac{6/l}{1+\alpha_y}(\zeta^2-\zeta)\\
R_{\psi_1}=\frac{1}{1+\alpha_y}(3\zeta^2-(4+\alpha_y)\zeta+1+\alpha_y)&R_{\psi_2}=\frac{1}{1+\alpha_y}(3\zeta^2-(2-\alpha_y)\zeta)\\
R_{v_1}=\frac{6/l}{1+\alpha_z}(\zeta^2-\zeta)&R_{v_2}=\frac{6/l}{1+\alpha_z}(-\zeta^2+\zeta)\\
R_{\theta_1}=\frac{1}{1+\alpha_z}(3\zeta^2-(4+\alpha_z)\zeta+1+\alpha_z)&R_{\theta_2}=\frac{1}{1+\alpha_z}(3\zeta^2-(2-\alpha_z)\zeta)
\end{array}\right.
\end{equation*}
where, $ \alpha_y=2K_y/l^2=\frac{12EI_y}{\kappa GAl^2} $, $ \alpha_z=2K_z/l^2=\frac{12EI_z}{\kappa GAl^2} $, \& $ \zeta=x/l $.
\ref{eq:InterpolationFunctions} is expressed in matrix form as it appears in \ref{eq:FEGoverning} where
\begin{equation}\label{ShapeFunctionMatrix}
\bunderline{\mathbf{N}}(x)=\left[\begin{array}{cccccccccccc}
T_{v_1}&0&0&T_{\theta_1}&T_{v_2}&0&0&T_{\theta_2}&0&0&0&0\\
0&T_{w_1}&T_{\psi_1}&0&0&T_{w_2}&T_{\psi_2}&0&0&0&0&0\\
0&R_{w_1}&R_{\psi_1}&0&0&R_{w_2}&R_{\psi_2}&0&0&0&0&0\\
R_{v_1}&0&0&R_{\theta_1}&R_{v_2}&0&0&R_{\theta_2}&0&0&0&0\\
0&0&0&0&0&0&0&0&N_1&0&N_2&0\\
0&0&0&0&0&0&0&0&0&N_1&0&N_2
\end{array}\right]
\end{equation}
\subsection{Beam Equations}
Substitute \ref{eq:FEGoverning} into \ref{eq:EquationOfMotionVirtual}
\begin{equation}\label{key}
\int_0^l \bunderline{\mathbf{N}}^\T\delta\vec{\mathbf{q}}^\T\bunderline{\mathfrak{M}}\bunderline{\mathbf{N}}\ddot{\vec{\mathbf{q}}}dx+\int_0^l \bunderline{\mathbf{N}}^\T\delta\vec{\mathbf{q}}^\T\bunderline{\mathfrak{G}}\bunderline{\mathbf{N}}\dot{\vec{\mathbf{q}}}dx+\int_0^l\bunderline{\mathbf{N}}^\T\delta\vec{\mathbf{q}}^\T\bunderline{\mathcal{B}}^\T\bunderline{\mathcal{D}}\bunderline{\mathcal{B}}\bunderline{\mathbf{N}}\vec{\mathbf{q}}dx=0
\end{equation}
note that now $ \vec{\mathbf{q}} $ is not dependent on x so it, and it's derivatives, may be pulled out of the integral. Also, a substitution of $ \bunderline{\mathbf{B}}=\bunderline{\mathcal{B}}\bunderline{\mathbf{N}} $ is made leading to the differential equation
\begin{equation}\label{eq:GoverningDifferentialEquationDiscrete}
\int_0^l \bunderline{\mathbf{N}}^\T\bunderline{\mathfrak{M}}\bunderline{\mathbf{N}}dx\ddot{\vec{\mathbf{q}}}+\int_0^l \bunderline{\mathbf{N}}^\T\bunderline{\mathfrak{G}}\bunderline{\mathbf{N}}dx\dot{\vec{\mathbf{q}}}+\int_0^l\bunderline{\mathbf{B}}^\T\bunderline{\mathcal{D}}\bunderline{\mathbf{B}}dx\vec{\mathbf{q}}=0
\end{equation}
Define 
\begin{subequations}\label{key}
\begin{empheq}[left={\empheqlbrace\,}]{align}
\bunderline{\mathbf{M}}&=\int_0^l \bunderline{\mathbf{N}}^\T\bunderline{\mathfrak{M}}\bunderline{\mathbf{N}}dx\label{eq:ConsistentMassMatrix}\\
\bunderline{\mathbf{G}}&=\int_0^l \bunderline{\mathbf{N}}^\T\bunderline{\mathfrak{G}}\bunderline{\mathbf{N}}dx\label{eq:ConsistentGyroMatrix}\\
\bunderline{\mathbf{K}}&=\int_0^l\bunderline{\mathbf{B}}^\T\bunderline{\mathcal{D}}\bunderline{\mathbf{B}}dx\label{eq:ConsistentStiffnessMatrix}
\end{empheq}
\end{subequations}
So that, the general equations of motion for the timoshenko beam element are
\begin{equation}\label{eq:EquationsOfMotionTimoElementGeneral}
\bunderline{\mathbf{M}}\ddot{\vec{\mathbf{q}}}+\bunderline{\mathbf{G}}\dot{\vec{\mathbf{q}}}+\bunderline{\mathbf{K}}\vec{\mathbf{q}}=0
\end{equation}
