\chapter{Frequency Domain Analysis}\label{FrequencyDomainAnalysis}
Eigenvalues and eigenvectors of equations of motion will be used to provide information on the modal content of the system being analyzed. Eigenvalues can provide information on the natural frequencies, damping, and stability of the modes of vibration. Eigenvectors will be used to describe the shape of each of these modes of vibration including relative phase information for each nodal location. Frequency response functions will describe amplitude of response, phase lag of response from forcing input, and further stability information on modes.\par
Rotordynamic analyses are often interested in the dependence of the system on the spin speed $ \Omega $. The inclusion of this parameter as a time dependent variable would make the solutions of the system much more difficult. In this work, the spin speed is considered to be constant in each operation, taken in a series of operations as the spin speed is changed. In this way, the effect of spin acceleration is not taken into account, but the dependence on spin speed is approximated. For the vast majority of rotordynamic systems this approximation is sufficiently accurate, especially when considering that it is not a very common neccessity to ramp up quickly in speed.\par
For a simple critical speed computation, assuming the form of a homogeneous solution $ \vec{\mathbf{q}}=e^{i\Omega t} $ while neglecting all damping and forcing in the equations of motion will provide the dynamic undamped forced whirling matrix. Exploring the eignevalues of this equation will give the undamped critical speeds of the system. In the case that the natural frequencies (whirl speeds) need to be calculated independent of spin speed, like in the Campbell diagram, a solution of the form $  \vec{\mathbf{q}}=e^{s t}$ is assumed so that the spin speed and whirl speed can vary independently. Eigenvalues of this dynamic free whirling matrix will provide complex numbers of which the real part is proportional to damping, and the imaginary is damped natural frequencies of the system. Particular integrals of the form $ \vect{q}=e^{i\omega t}\ \&\ \vect{q}=e^{i\Omega t} $ will be assumed to find the frequency response.\par 
Details of the different analyses to follow will be accompanied by an example problem to demonstrate the results. The problem of interest is a simple two disk rotor system depicted in Figure \ref{fig:TwoDiskDiagram}. The geometry and material properties are listed in Table \ref{tab:ExampleRotorTable}. The rotor is discretized as in figure \ref{fig:TwoDiskDiagram} into 6 elements with geometry of $ L=1.8[m] $, $ a=.3[m] $, \& $ b=.6[m] $ and bearings located at the ends of the shaft.\par 
\begin{figure}
	\centering
	\def\svgwidth{400pt}
	\import{figures/}{TwoDiskDiagram.pdf_tex}
	\caption{Diagram of Two disk model example problem.}
	\label{fig:TwoDiskDiagram}
\end{figure}
\begin{table}
\caption{Properties of disks, shaft elements, and bearings of the example problem.} \label{tab:ExampleRotorTable}
\centering
\begin{tabular}{rcccccc}
						&$\rho[\frac{kg}{m^3}]$	&$d_s[m]$					&$\nu$				&$E[Pa]$			&$ \eta_v[s] $	&$ \eta_h $	\\\hline
	\textbf{Shaft}		&$7850$					&$0.1$					&$0.3$				&$\num{210E9}$		&$ 0.0002 $		&$ 0 $		\\[1em]
						&$\rho[\frac{kg}{m^3}]$	&$d_d[m]$					&$l_d[m]$				&					&				&			\\\hline
	\textbf{Disks;} $ \mathbf{D_1,\ D_2} $		&$7850$					&$0.6$					&$0.1$				&					&				&			\\[1em]
						&$k_x[\frac{N}{m}]$		&$k_y[\frac{N}{m}]$		&$c_x[\frac{Ns}{m}]$&$c_y[\frac{Ns}{m}]$&				&			\\\hline
	\textbf{Bearings;} $ \mathbf{B_1,\ B_2} $	&$\num{1e7}$			&$\num{1e7}$			&$100$				&$100$				&				&			
\end{tabular}
\centering
\end{table}
\section{State Space Representation and the Eigenvalue Problem}
The system can be represented in state space where,
\begin{equation}\label{eq:StateSpaceDef}
\dot{\vec{\mathbf{z}}}=\bunderline{\mathbf{A}}\vec{\mathbf{z}}+\bunderline{\mathbf{B}}\vec{\mathbf{w}},\quad\vec{\mathbf{z}}=\left\{\begin{array}{c}\dot{\vec{\mathbf{q}}}\\\vec{\mathbf{q}}\end{array}\right\}
\end{equation}
$ \bunderline{\mathbf{B}}\vec{\mathbf{w}} $ is an input into the system, which here will represent the forces due to unbalance. Solving for $ \bunderline{\mathbf{A}}\  \&\   \bunderline{\mathbf{B}}\vec{\mathbf{w}}$
\begin{equation}\label{key}
\bunderline{\mathbf{A}}=\left[\begin{array}{cc}
-\bunderline{\mathbf{M}}^{-1}\bunderline{\mathbf{D}}&-\bunderline{\mathbf{M}}^{-1}\bunderline{\mathbf{K}}\\
\bunderline{\mathbf{I}}&\bunderline{\mathbf{0}}
\end{array}\right]\quad\&\quad\bunderline{\mathbf{B}}\vec{\mathbf{w}}=\Omega^2\left\{\begin{array}{c}
\mat{M}^{-1}\vec{\mathbf{F}}\\0\end{array}\right\}
\end{equation}
This equation is valid in both the complex and real coordinate plane. With the use of complex coordinates, the state vector is represented by $ \vect{z}=[\dot{\vect{q}}^c,\vect{q}^c]^\T $. $ \bunderline{\mathbf{I}}\ \&\ \bunderline{\mathbf{0}} $ are appropriately sized to match the matrices in the system of choice. $ \mat{A} $ is the dynamic matrix of the system, which represents the dynamics in a single expression. Now in the form of a first order linear differential equation, numerical integration and many other numerical analysis techniques may be applied.\par
Using equation \eqref{eq:StateSpaceDef} and assuming a solution of $ \vect{z}=\vect{\Theta}e^{st} $ while neglecting the forcing term $ \mat{B}\vect{w} $ leads to the eigenvalue problem
\begin{equation}\label{eq:StateSpaceEigenValueProblem}
(\mat{A}-s)\vect{\Theta}=\mat{0}
\end{equation} 
$ \vect{\Theta} $ is the vector containing the eigenvectors. Since the state in which these equations are defined is the combination of displacement and velocity, the eigenvectors are defined as a set corresponding to both displacement and velocity. This effectively duplicates the set as
\begin{equation*}
\vect{\Theta}=\left\{\begin{array}{c}\vect{\theta}_{\dot{\vect{q}}}\\\vect{\theta}_{\vect{q}}\end{array}\right\}=\left\{\begin{array}{c}s\vect{\theta}_{\vect{q}}\\\vect{\theta}_{\vect{q}}\end{array}\right\}
\end{equation*}

\section{Dynamic Response}
The dynamic response of the system to unbalance can be achieved by substituting the particular integral $ \vect{q}=\vect{q}_0e^{i\Omega t} $ into the equation of motion \eqref{eq:GlobalSystemofEquationsReal},\eqref{eq:GlobalSystemofEquationsComplex}. 
\begin{equation}
\left\{\begin{array}{rl}
\vect{q}_0&=(-\Omega^2\mat{M}+i\Omega\mat{D}+\mat{K})^{-1}\Omega^2\vect{F}\\
\vect{q}_0&=H(\Omega)
\end{array}\right.
\end{equation}
where $ H(\omega) $ is called the complex frequency response of the system due to unbalance. The frequency response acts as a transfer matrix that converts inputs, $ \vect{F} $, into outputs $ \vect{q}_0 $. Use of this frequency response function with real coordinates requires a multiplication of the complex unit $ i $ to one of the orthogonal directions to ensure correct phase information. For the coordinate system set in \S\ref{fig:TimoBeamDOF} this results in $ i $ times $ w $ and $ i $ times $ \psi $. This multiplication reflects the fact that $ w $ lags $ v $ by 90 degrees and $ \psi $ lags $ \theta $ by 90 degrees in reference to a counterclockwise $ \Omega $. Values of $ H $ are complex with the absolute part representing the magnitude of the response, and the angle representing the phase delay of the response from the input. If the input phase is known in terms of shaft angle, then the phase delay difference of the response angle from the shaft angle of the input can be interpreted as a shaft angle output.\par 
The response can also be calculated using the particular integral $ \vect{q}=\vect{q}_0e^{i\omega t} $ where $ \omega $ is the independent whirl frequency. Then the response is found as the transfer response of the arbitrary oscillation, while spin speed remains independent.\par
For the example problem defined in \S\ref{FrequencyDomainAnalysis} the unbalance response to a disk b unbalance of $ \num{2e-5}[m] $ is shown in Figure \ref{fig:ExampleBode}. Note that the phase lag angle and the amplitude of the response are plotted side by side forming the Bode diagram. For this same set of data, real and imaginary parts of the complex frequency response may be plotted on the complex plane to form the Nyquist diagram. Figure \ref{fig:ExampleNyquist} contains a seperate diagram for each mode. During a natural frequency, if one of the circles traverses in the counterclockwise direction that mode is deemed unstable. Also, if the path of one circle crosses the path of the other pertaining to the orthogonal direction, the whirl is in the negative direction. 
\begin{figure}
	\def\width{\linewidth/1.5}
	\def\height{.25\linewidth}
	\def\sep{2.5cm}
	\pgfplotsset{
		every picture/.style={trim axis left,trim axis right},
		every axis/.style={
			width=\width,
			height=\height,
			xlabel={Speed, $ \Omega $[RPM]},
			xlabel style={yshift=0},
			ylabel={Phase, $ \beta $[deg]},
			ylabel style={xshift=0},
		},
		ticklabel style={
			/pgf/number format/fixed,
			/pgf/number format/precision=3
		},
	}
	\centering
	\import{}{./figures/ExampleBode.tex}
	\caption{Bode diagram of the second disk subject to an unbalance at the second disk. Amplitudes of $ \mathbf{v\ \&\ w} $ are identical for this antisymmetric system, phase of $ \mathbf{w} $ would be lagging $ \mathbf{v} $ by $ \mathbf{90[deg]} $.}
	\label{fig:ExampleBode}
\end{figure}
\begin{figure}
	\begin{subfigure}{\textwidth}
	\def\width{.4\linewidth}
	\def\sep{2.5cm}
	\pgfplotsset{
		every picture/.style={trim axis left,trim axis right,trim axis bottom},
					every axis/.style={
			zlabel style={yshift=-1cm,xshift=1.8cm,rotate=-90},
			zticklabel style={right,},
		},
	}
	\centering
	\import{}{./figures/ExampleNyquist1.tex}
	\caption{Nyquist Diagram for the second mode in the speed range $ 1180 < \Omega < 1195[RPM] $.}
	\end{subfigure}
	\begin{subfigure}{\textwidth}
			\def\width{.4\linewidth}
		\def\sep{2.5cm}
		\pgfplotsset{
			every picture/.style={trim axis left,trim axis right,trim axis bottom},
			every axis/.style={
				zlabel style={yshift=-1cm,xshift=2cm,rotate=-90},
				zticklabel style={right,xshift=2pt},
			},
		}
		\centering
		\import{}{./figures/ExampleNyquist2.tex}
		\caption{Nyquist Diagram for the second mode in the speed range $ 5100 < \Omega < 5300[RPM] $.}
	\end{subfigure}
	\caption{Nyquist plots for the first two modes at node 6 for the example two disk problem.}
	\label{fig:ExampleNyquist}
\end{figure}
Using the particular integral of $ \vect{q}=\vect{q}_0e^{i\omega t} $ instead of the synchronous solution, provides whirl frequencies independent of spin speed. This allows for the investigation of all modes of vibration at a specific spin speed. For instance, if the spin speed is set to $ 4000[RPM] $ and the Nyquist plot for the first mode is revisited (Figure \ref{fig:ExampleNyquistIndependent1}) it is evident that the path traced is in the counter-clockwise direction indicating instability of mode 1 at the spin speed of $ 4000[RPM] $. 
\begin{figure}
	\def\width{.4\linewidth}
	\def\sep{2.5cm}
	\pgfplotsset{
		every picture/.style={trim axis left,trim axis right,trim axis bottom},
		every axis/.style={
			zlabel style={yshift=-1cm,xshift=2cm,rotate=-90},
			zticklabel style={left,xshift=-2pt},
			},
	}
	\centering
	\import{}{./figures/ExampleNyquistIndependent1.tex}
	\caption{Nyquist Diagram for the first mode at node 6 in the whirl speed range $ 1100 < \omega < 1300[RPM] $ at a spin speed, $ \Omega=4000[RPM] $. Counter-clockwise path indicates instability.}
\end{figure}
\section{Roots Locus and Stability Analysis}
One valuable facet of a rotordynamic model is its ability to predict instability. Unstable operation of rotor systems can lead to failures and unsafe operating conditions. Before the advent of modern predictive models for rotating machinery, it was not common to operate a machine above the first critical speed. Internal damping and other rotating damping can cause a subsynchronous whirl when operating above the first critical speed, this would often lead to failure of machines. The rotating damping effect is worse for machines with stiff bending modes, and is often instigated by loose fittings, shrink fits, and couplings. As mentioned in \S\ref{Rotating Internal Damping}, the effect of external rotating damping sources will not be investigated here. Rather, the stability will be tested by modeling the internal damping of the rotor due to viscous heat production during loading and unloading of the beam in bending. See \S\ref{Rotating Internal Damping} for a more detailed explanation and derivation.\par 
Roots Locus is the plot of eigenvalues on the $ \Re-\Im $ plane as some value they are dependent on changes. In rotordynamic analysis the dependent variable is often the spin speed $ \Omega $. Using the definition of the state space eigenvalue problem set out in \eqref{eq:StateSpaceEigenValueProblem}, the eigenvalues $ s $ represent a complex set of which the real part is proportional to damping and the imaginary is the damped natural frequency. The roots locus is useful in determining stability of the system, and which mode is responsible for the instability. Using the eigenvalues $ s $ of the eigenvalue problem defined in equation \eqref{eq:StateSpaceEigenValueProblem} on the example problem defined in \S\ref{FrequencyDomainAnalysis} a plot of the roots locus can be found as in figure \ref{fig:ExampleRootsLocus}. Internal damping coefficient, $ \eta_v $, is added to the system at the value of $ 0.0002[s] $. The first three modes are presented with the negative complex eigenvalue being represented in the positive imaginary axis alongside each positive mode of vibration. In completely symmetric systems, positive and negative modes have identical eigenvalues at a spin speed of zero. As the spin speed increases, positive modes move in the positive real direction--becoming more unstable. On the other hand, the negative frequencies move in the negative real direction--becoming more stable. This phenomena of the positive modes becoming more unstable and the negative becoming more stable is a general trend, but not a rule, rotating damping defined in \S\ref{Rotating Internal Damping} assists in this trend. It is evident by looking at figure \ref{fig:ExampleRootsLocus} that the system becomes unstable at some point, as many of the eigenvalues are in the positive real region of the plane. According to the plot, the first and second modes both become unstable as speed increases. Gyroscopic effects tend to increase the whirl frequency (imaginary part of eigenvalue) of the positive mode, while the opposite effect is seen in the negative mode. In general, the effect of non-rotating damping is to rotate modes in the counter-clockwise direction, at first away from the imaginary axis, and eventually on to the real axis when the mode becomes over-damped. The goal of the application of the magnetic bearing in \S\ref{MagExample} will be to provide enough non-rotating damping to render as many modes as possible over-damped.\par 
\begin{figure}
	\def\width{.6*\linewidth}
	\def\height{.3*\linewidth}
	\def\sep{3em}
		\pgfplotsset{
	every picture/.style={trim axis left,trim axis right},
	every axis/.style={
		width=\width,
		height=\height,
		xlabel={$ \Re(s) $},
		xlabel style={yshift=0},
		ylabel={$ \Im(s)[Rad/s] $},
		ylabel style={xshift=0},
	},
	ticklabel style={
		/pgf/number format/fixed,
		/pgf/number format/precision=3
	},
}
	\centering
	\import{}{./figures/ExampleRootsLocus.tex}
	\caption{Roots Locus of the example problem with an internal damping coefficient of $ 0.0002[s] $.}
	\label{fig:ExampleRootsLocus}
\end{figure}
The Roots Locus for the example problem is also presented in three dimensions to lend in the understanding of the relationship with spin speed. This can be seen in Figure \ref{fig:ExampleRootsLocus3D}.
\begin{figure}
	\def\width{.6*\linewidth}
	\def\height{.3*\linewidth}
	\def\sep{3em}
	\pgfplotsset{every picture/.style={trim axis left, trim axis right}, every axis/.style={title style={yshift=-.8em}}, grid style={line width=.1pt, draw=gray!10},major grid style={line width=.2pt,draw=gray!50},minor tick style={draw=none}}
	\centering
	\import{figures/}{ExampleRootsLocus3D.tex}
	\caption{Roots Locus of the example problem in 3-D with an internal damping coefficient of $ 0.0002[s] $.}
	\label{fig:ExampleRootsLocus3D}
\end{figure}
A direct method of measuring the stability of the system is to observe only the real part of the eigenvalues of the dynamic matrix as they vary with spin speed. Since the solution was assumed the be of the form $ \vect{z}=\vect{\Theta}e^{st} $, then the eigenvalues represent the values of the complex exponent $ s $. Splitting $ s $ into its real and complex components as $ s=\sigma+i\omega_{d} $, and plugging into $ \vect{z} $ gives: $ \vect{z}=\vect{\Theta}e^{\sigma+i\omega_{d}} $. By inspection, it is evident that when the real part is positive, the response will grow without bound--an unstable system.\par
Damping is commonly represented as a the ratio of the real part of the eigenvalue to the natural frequency as
\begin{equation}
\zeta=\frac{-\sigma}{\omega_n}\ ,\ \omega_n=\sqrt{\sigma^2+\omega_{d}^2}
\end{equation}
where $ \zeta $ is the damping ratio, and  $ \omega_n $ is the natural frequency. The stability margin, or the range of speeds through which the system remains stable can be determined by plotting the maximum real part, $ \Re(s) $, of the set against spin speed. The point at which $ \Re(s) $, or $ \sigma $, crosses the real axis is the threshold of stability. The plot which represents this is termed the Stability Margin plot, and is shown in \ref{fig:ExampleStabilityMargin} for the example problem defined in \S\ref{FrequencyDomainAnalysis}.\par
\begin{figure}
	\def\width{.6*\linewidth}
	\def\height{.4*\linewidth}
	\def\sep{3em}
	\pgfplotsset{
		every picture/.style={trim axis left,trim axis right},
		every axis/.style={
			width=\width,
			height=\height,
			xlabel={Speed, $ \Omega $[RPM]},
			xlabel style={yshift=3em},
			ylabel={Maximum $ \Re(s) $},
			ylabel style={xshift=2em},
			scaled y ticks = false,
		},
		ticklabel style={
			/pgf/number format/fixed,
			/pgf/number format/precision=3
		},
	}
	\centering
	\import{figures/}{StabilityMargin.tex}
	\caption{Stability plot of the example problem.}
	\label{fig:ExampleStabilityMargin}
\end{figure}
Finally, the stability can be analyzed using the damping coefficients($ \zeta $) plotted against speed. This figure, shown in \ref{fig:ExampleDamping}, indicates instability as when $ \zeta $ drops below zero.
\begin{figure}[!htb]
	\def\width{.7\linewidth}
	\def\height{.5\linewidth}
	\def\sep{3em}
	\pgfplotsset{
		every picture/.style={trim axis left,trim axis right},
		every axis/.style={
			width=\width,
			height=\height,
			xlabel={Speed, $ \Omega $[RPM]},
			xlabel style={yshift=3em},
			ylabel={Damping Ratio, $\zeta$},
			ylabel style={xshift=2em},
			scaled y ticks = false,
		},
		ticklabel style={
			/pgf/number format/fixed,
			/pgf/number format/precision=3
		},
	}
	\centering
	\import{figures/}{ExampleDamping.tex}
	\caption{Damping ratio of the first three modes of the example problem, with indication of threshold of stability for each mode.}
	\label{fig:ExampleDamping}
\end{figure}
\section{Campbell}
The Campbell diagram correlates the spin speed and the whirl speed. Whirl speed is calculated as the imaginary part of the complex eigenvalue defined by the eigenvalue problem in equation \ref{eq:StateSpaceEigenValueProblem}. Spin speed is varied while calculating all of the critical whirl speeds at each step of spin speed. The campbell diagram for the example problem is given in figure \ref{fig:ExampleCampbell}. This diagram can be used to discern the critical speeds of the system by inspecting where the $ \omega=\Omega $ synchronous line passes through positive and negative modes in this speed range. Here the synchronous line passes through three separate modes, each with a positive and negative whirl speed. Note: with the large influence of the gyroscopic effect in the example problem presented here, the positive and negative whirl speeds for the second and third modes diverge from one another rather quickly, but this is not the case for the first mode. A physical interpretation of this phenomena will be presented in \S\ref{Shapes} 
\begin{figure}[!htb]
	\def\width{.7*\linewidth}
	\def\height{.4*\linewidth}
	\def\sep{3em}
	\pgfplotsset{every picture/.style={trim axis left, trim axis right}, every axis/.style={ylabel style={yshift=2pt},xlabel style={yshift=0}}}%, every axis/.style={hide axis}}%
	\centering
	\import{figures/}{ExampleCampbell.tex}
	\caption{Campbell Diagram of the example problem.}
	\label{fig:ExampleCampbell}
\end{figure}
\section{Shapes}\label{Shapes}
Deformed shape of each mode can be predicted using the eigenvectors of the eigenvalue problem from equation \ref{eq:StateSpaceEigenValueProblem}. Eigenvectors contain displacement arrangement that corresponds to a natural frequency; this includes phase and relative amplitude of the generalized displacements from one another. A simple model will be utilized to conceptualize this idea, consider a rigid rod that can only translate in one direction at each end. The eigenvectors will contain all linearly independent combinations of the movement of the left side relative to the right. This would be in general; opposite left and right (i.e. one up, one down), and same left and right(i.e. both up, or both down). The bending modes of the beam operate in a similar manner but scaled up, and just like in the simple case, including the same number of modes as there are degrees of freedom in the system. Since the state space representation used for eigenanalysis contains $ \dot{\vect{q}},\; \& \;\vect{q} $ the eigenvector will contain $ 2N $ modes. For a general $ i $th mode of vibration, the eigenvector $ \vect{\Theta}_i $ contains the displacement information. But, since the portion of $ \vect{\Theta}_i $ pertaining to velocities, $ \vect{\theta}_{\dot{\vect{q}}i} $, is a linear combination of $ \vect{\theta}_{\vect{q}i} $, all of the information is contained in one of these arrays. Therefore, the interpolation of displacements for the $ i $th mode is given by
	\begin{equation}
	\vect{u}_i=\mat{N}\vect{\theta}_{\vect{q}i}
	\end{equation}
where, the value of $ \vect{u}_i $ is now a function of $ x $. So, through the choice of an array of positions that span the length of the beam element the displacement, and phase angle distribution of the $ i $th mode of vibration is determined. Amplitude in eah direction is taken as the absolute value of the corresponding eigenvector value for that node and that degree of freedom. Phase angle is obtained from the angle of that same complex value. Phase angle and amplitude of an transverse pair may be used to create an orbit of a beam axis location $ x $. Then all orbits along the beam axis can be stacked and a shape is formed as in figure \ref{fig:ExampleShape}.\par
Shape figures for the first 3 modes of the example problem are presented in Figures \ref{fig:ExampleShape1},\ref{fig:ExampleShape2}, and \ref{fig:ExampleShape3}. Note that the real and imaginary parts of the eigenvalues are listed along with the shapes. Also, note that the speed is chosen to coincide with the damped natural frequency so that this is the critical speed bending shape. Shapes of beam modes may be determined for any spin speed $ \Omega $.\par
\begin{figure}[h!]
	\def\cs{.315}	
	\def\overlap{1em}
	\pgfplotsset{every picture/.style={trim axis left, trim axis right}, every axis/.style={yticklabel style={xshift=.4em,yshift=.4em},minor tick num=2}, grid style={line width=.1pt, draw=gray!10},major grid style={line width=.2pt,draw=gray!50},ticks=none,minor tick style={draw=none}}%, every axis/.style={hide axis}}%
\begin{subfigure}{\cs\textwidth}
	\centering
	\def\width{\linewidth+\overlap}
	\def\height{\linewidth}
	\import{figures/}{ExampleShape1.tex}
	\caption{Mode Shape 1.}
	\label{fig:ExampleShape1}
\end{subfigure}
\begin{subfigure}{\cs\textwidth}
	\centering
	\def\width{\linewidth+\overlap}
	\def\height{\linewidth}
	\import{figures/}{ExampleShape2.tex}
	\caption{Mode Shape 2.}
	\label{fig:ExampleShape2}
\end{subfigure}
\begin{subfigure}{\cs\textwidth}
	\centering
	\def\width{\linewidth+\overlap}
	\def\height{\linewidth}
	\import{figures/}{ExampleShape3.tex}
	\caption{Mode Shape 3.}
	\label{fig:ExampleShape3}
\end{subfigure}
\caption{Modal shapes of the example problem.}\label{fig:ExampleShape}
\end{figure}
 A physical interpretation of the selective effect gyroscopic moments have on the whirl speeds of different mode shapes can be presented by taking the mode shapes into account. Recall that the disks in the example are mounted at nodal locations 2 \& 3. In figure \ref{fig:ExampleShape} it is evident that in the first mode, Figure \ref{fig:ExampleShape1}, both disks are translating together. As a matter of fact, most of all the points of the system are translating in phase. This type of mode is called a cylindrical mode for the fact that as the shaft's path is traced, as it is in \ref{fig:ExampleShape1}, it forms a cylinder. The consequence of this type of motion is small transverse rotation angles, $ \psi\ \&\ \theta $, of the mass elements. Since the gyroscopic moments are proportional to these angles, their magnitude is very low compared to transverse effects and as a result the whirl speeds are nearly unaffected by spin speed changes. On the contrary, modes 2 \& 3 of figures \ref{fig:ExampleShape2} \& \ref{fig:ExampleShape3} respectively, the angle of the disks and other mass elements is changing significantly through rotation. Because of this, large gyroscopic moments induce an out of phase force that stiffens positive whirl and softens negative whirl tendencies. Gyroscopic moments are larger in these types of modes where there is a point of inflection on the beam, commonly called an antinode of vibration. This mode is called a conical mode because the shape of the path forms one or many cones.\par