\chapter{Conclusion}\label{conclusion}
\section{Summary}
The aim of this work was to provide a basis for development of rotordynamic models, and their comparison to experimental results. A finite element method designed specifically for creation of these models was presented. Techniques for analyzing models in the frequency domain were explained and the results demonstrated. Methods for also analyzing experimental vibration signals provided a correlation between models and experimental results. Correlations between experimental results and theoretical models were explored in the optimization of an overhung rotor levitated by an active magnetic bearing. Interpretations of both model and experimental signal analysis were provided to lend in understanding results. Resulting analyses and interpretations can be used to identify problems with existing rotating machines as well as aid in the design process of new rotating machines.\par
\section{Future Work}
This work is intended to be a building block on which future students at Cal Poly may build more advanced models and signal processing techniques in the field of rotordynamics. Possible future projects related to the extension of this work are:
\begin{itemize}
	\item an experiment with the application of an active magnetic bearing on an overhung rotor system, using the parameters and methods determined in this work
	\item extension of this finite element model to include non-linear internal damping constitutive relationships
	\item extension of this finite element model to include a more detailed disk model for the analysis of turbomachines
	\item extension of this finite element model to include damping effects from various fittings and couplings
\end{itemize}