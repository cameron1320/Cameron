\chapter{Introduction}
\section{Background} 
The study of rotating machines is a relatively new study of physics beginning in the late 1800 with the industrial revolution driving to the forefront a need to mitigate dangerous operating conditions of machines such as the steam turbine. Increasing demands on power generation were leading to increased size and speed of such turbines, which in turn increased the need for understanding of the vibrations they produced. Engineers and physicists alike thought it was impossible to operate continuously at a speed above the first natural frequency. One of the first papers published on the matter by Rankine in 1869 even proposed this to be fact, that rotors could not be stable above this first natural frequency. Jeffcott produced one of the first meaningful solutions to the rotrodynamic problem in 1919 with his model that is now known as the Jeffcott rotor. This rotor is a simple model with a single disk located equidistant from two bearing supports. In his paper he made the realization that after the first natural frequency, a rotor in this configuration is self centering (begins to rotate abou its center of mass). This realization led to the operation of many machines in the so-called supercritical speed range, and increased power and efficiency of many industrial machines.\par 
Today our understanding of rotordynamics has led to the operation of some  machines for tooling, for example, that spin at speeds of over a hundred thousand revolutions per minute--many times the first natural frequency. The increase in use of computers in the study of rotordynamics has led many innovations allowing new rotors to push these boundaries. Additionally, increasing use of active control devices such as the Active Magnetic Bearing (AMB) allows for adaptive control of rotating machines.\par 
Rotors being developed today are becoming increasingly complex, as our understanding is allowing them to become. The trend with new systems is driving the analysis of rotordynamics as part of the design process instead of balancing an already built system. The Finite Element Method (FEM) which is widely used in many other fields such as structures and impact analysis, has spread in its use for rotordynamic analysis as part of design. With FEM, the design geometry can be parametized and systems can be optimized in terms of geometry and material use to minimize vibration. Though readily available FEM packages can be altered for use in rotordynamics, it is often the case that elements available are not designed with the inclusion of gyroscopic moments. Gyroscopic moments provide a vital part aspect of the physics of rotordynamc systems and even more so in high speed systems.\par 
There exists an array of FEM packages to specifically address the rotordynamic model such as: DyRoBeS, DYNROT, and XLRotor. These packages are able to provide valuable design information on complex linear and non-linear FEM models. There also exists many solutions for experimental analysis and monitoring of rotating machines. Bently Nevada has been the leading producer of such monitoring equipment for rotordynamics for several decades, with software and hardware solutions that produce monitoring, diagnostics, and visualizations of rotor vibrations.\par
\section{Literature Review}
Nelson \& McVaugh \cite{nelson1976dynamics} introduced an early Finite Element for use in rotordynamics that included gyroscopic moments in 1976. This model used a Bernoulli-Euler beam to define the internal bending stiffness of the element, with cubic spline shape interpolation functions. The Bernoulli-Euler beam does not include shear effects limiting its use to relatively slender beams. Nelson made an improvement to this model in 1977 with Zorzi \cite{zorzi1977finite} with the inclusion of internal damping in the constitutive relationship in the form of viscous and hyteretic damping mechanism. Nelson (1980)\cite{nelson1980finite} then extended his 1976 model to include shear deformation of the beam element, known as the Timoshenko beam element. But, did not include internal viscous or hysteretic damping in the model. Ozguven, in 1984\cite{ozguven1984whirl}, extended Zorzi's 1980 Timoshenko model to include shear deformation effects.  An excelent derivation of the Timoshenko beam element is provided by Luo (2008)\cite{luo2008efficient}. His derivation results in the static homogeneous Euler-Lagrangian differential equations that pertain to bending, axial, and torsional motion. Additionaly, Luo provides a derivation of the shape interpolation functions that reduce shear locking in the Timoshenko beam element. All of these beam models assume a homogeneous beam for each element. Derivations of beam models exist that do not assume a homogeneous beam such as the paper on conical beam elements by Greenhill, Bickford, and Nelson in 1985\cite{greenhill1985conical}, and the paper on consistent beams from Genta in 1985\cite{genta1985consistent}. This non-homogeneous extension is not realized in the model presented in this work. Genta's textbook on rotordynamics, published in 2007\cite{genta2007dynamics} involves a detailed derivation of the finite element equations using the Timoshenko beam and shear-lock reducing shape interpolation functions.\par 
Important anecdotal notes on understanding of the rotating damping effect are presented in Kandil's 2004\cite{kandil2005rotor} thesis ``On Rotor Internal Damping Instability.'' Kandil presents a strong physical description of the internal damping effect and the many sources it has. Further interpretations for the effect of internal damping were presented by Genta in his paper ``On a Persistent Misunderstanding of the Role of Hysteretic Damping in Rotordynamics.'' Where the common misunderstanding of sub-critical instability due to hysteretic damping is rectified.\par 
The use of Bernoulli-Euler beam elements in the application of a vibration reducing Active Magnetic Bearing (AMB) is presented in the paper from Das, et al. (2008\cite{das2008vibration}). Das also provides the derivation of a simple control algorithm for the AMB using a linearized magnetic force equation.\par
\section{Scope of Work}
This work will be comprised 
\begin{itemize}
	\item Methods for analyzing time domain vibration signals to produce information on frequency content, amplitude, and phase lag. Discussion of where these time domain signals come from and the requirements for accurate system characterization.
	\item Construction of a finite element model specifically formulated for use in rotating beams.
	\item Frequency domain analysis of rotordynamic models for the assessment of critical speeds, mode shapes, stability and damping characteristics.
	\item Combine methods for analyzing time domain vibration signals from experimental data and frequency domain analysis techniques of finite element models to optimize the performance of an active magnetic bearing supporting an overhung disk rotor system.
\end{itemize}
This work will not explore deeply into time domain signal processing, frequency domain analysis, or the finite element method, but rather will intend to provide an overview of these methods and how they pertain to rotating machinery. It is the hope that this work will be continued at Cal Poly in the future to allow students to explore more deeply into rotorsynamic topics.\par