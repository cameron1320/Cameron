\chapter{Introduction}
\section{Background} 
The study of rotating machines is a relatively new branch of physics, beginning in the late 1800's with the industrial revolution catalyzing a need to mitigate dangerous operating conditions of machines, such as the steam turbine. Increasing demands on power generation were leading to increased size and speed of such turbines, which in turn increased the need for understanding the vibrations they produced. Engineers and physicists alike thought it was impossible to operate continuously at a speed above the first natural frequency. One of the first papers published on the matter by Rankine in 1869 even posited this as fact--that rotors could not be stable above this first natural frequency. Jeffcott produced one of the first meaningful solutions to the rotrodynamic problem in 1919 with his model that is now known as the Jeffcott rotor. This rotor is a simple model with a single disk located equidistant from two bearing supports. In his paper he articulates his realization that after the first natural frequency, a rotor in this configuration is self centering (begins to rotate about its center of mass). This realization motivated engineers to design systems that could withstand forces involved with passing through the first natural frequency and operate in the supercritical range. Operating in this supercritical range allowed machines to operate more efficiently, and effectively.\par 
Today our understanding of rotordynamics has led to the operation of  machines that spin at speeds of over a hundred thousand revolutions per minute--many times the first natural frequency. The incorporation of computers in to the study of rotordynamics has accelerated innovation, allowing new rotors to push even these boundaries. Additionally, increasing use of active control devices such as the Active Magnetic Bearing (AMB) allows for adaptive control of rotating machines. This permits adaptability after design and adjustment to varying operating conditions. Applications today are increasingly demanding. \par
Rotors developed today are increasingly complex. The trend in new applications is to incorporate rotordynamic analysis in the design process, rather than fixing already built systems. As rotordynamic analysis becomes increasingly integral in the design process, the Finite Element Method (FEM) is proving to be a vital design tool. FEM is widely used in many other fields, such as structures and impact analysis. With FEM, design parameters can be optimized in terms of geometry and material properties to minimize vibration. Though popular FEM packages can be altered for use in rotordynamics, it is often the case the elements included are not designed without consideration of gyroscopic moments. Gyroscopic moments fundamentally change the dynamics of a rotating sytem, especially at high speeds. Thus, their inclusion in FEM is vitally important to produce accurate results.\par 
An array of FEM packages exist to specifically design rotordynamic models that meet the complexity demanded by modern applications (including gyrospcopic moments) such as: DyRoBeS, DYNROT, and XLRotor. These packages provide valuable design information on complex linear and non-linear FEM models. Bently Nevada has been the leading producer monitoring equipment for rotordynamics for several decades, with software and hardware solutions that produce solutions for experimental analysis and monitoring of rotating machines, diagnostics, and visualizations of rotor vibrations.\par
\section{Literature Review}\label{LiteratureReview}
Nelson \& McVaugh \cite{nelson1976dynamics} introduced an early Finite Element for use in rotordynamics that included gyroscopic moments in 1976. This model used a Bernoulli-Euler beam to define the internal bending stiffness of the beam element. The Bernoulli-Euler beam does not include shear effects limiting its use to relatively slender beams. Nelson made an improvement to this model in 1977 with Zorzi \cite{zorzi1977finite} with the inclusion of rotating internal damping in the constitutive relationship in the form of viscous and hyteretic damping mechanism. Nelson (1980)\cite{nelson1980finite} then extended his 1976 model to include shear deformation of the beam element, known as the Timoshenko beam element. But, did not include internal viscous or hysteretic damping in the model. Ozguven, in 1984\cite{ozguven1984whirl}, extended Zorzi's 1980 Timoshenko model to include shear deformation effects.  An excelent derivation of the Timoshenko beam element is provided by Luo (2008)\cite{luo2008efficient}. His derivation results in the static homogeneous Euler-Lagrangian differential equations pertaining to bending, axial, and torsional motion. Additionaly, Luo provides a derivation of the shape interpolation functions that reduce shear locking in the Timoshenko beam element. All of these beam models assume a homogeneous beam for each element. Derivations of beam models exist that do not assume a homogeneous beam, such as the paper on conical beam elements by Greenhill, Bickford, and Nelson in 1985\cite{greenhill1985conical}, and the paper on consistent beams from Genta in 1985\cite{genta1985consistent}. This non-homogeneous extension is not realized in the model presented in this work. Genta's textbook on rotordynamics, published in 2007\cite{genta2007dynamics} involves a detailed derivation of the finite element equations using the Timoshenko beam and shear-lock reducing shape interpolation functions.\par 
Important anecdotal notes on understanding of the rotating damping effect are presented in Kandil's 2004\cite{kandil2005rotor} thesis ``On Rotor Internal Damping Instability.'' Kandil presents a strong physical description of the internal damping effect and the many sources it has. Further interpretations for the effect of internal damping were presented by Genta in his paper ``On a Persistent Misunderstanding of the Role of Hysteretic Damping in Rotordynamics,'' where the common misunderstanding of sub-critical instability due to hysteretic damping is rectified.\par 
The use of Bernoulli-Euler beam elements in the application of a vibration reducing Active Magnetic Bearing (AMB) is presented in the paper from Das, et al. (2008\cite{das2008vibration}). Das also provides the derivation of a simple control algorithm for the AMB using a linearized magnetic force equation.\par
Frequency domain analysis of rotordynamic models specifically is discussed in Genta's 2007 textbook ``Dynamics of rotating systems,''\cite{genta2007dynamics}. Additional overview of frequency domain analysis for vibrations in general is provided by Craig in the 2006 textbook ``Fundamentals of structural dynamics,''\cite{craig2006fundamentals}. Time domain analysis of vibration signals is also discussed in this text. Agnes Muszynska lends understanding for the concept of complex spectrum, and interpretations specific for rotordynamics in her 2005 ``Rotordynamics'' textbook, \cite{muszynska2005rotordynamics}, and her paper published with Goldman in 1999 on the matter \cite{goldman1999application}.
\section{Scope of Work}
This work will be comprised of: 
\begin{itemize}
	\item Methods for analyzing time domain vibration signals to produce information on frequency content, amplitude, and phase lag, as well as discussion of where these time domain signals come from and the requirements for accurate system characterization.
	\item Construction of a finite element model specifically formulated for use in rotating beams.
	\item Frequency domain analysis of rotordynamic models for the assessment of critical speeds, mode shapes, stability, and damping characteristics.
	\item time domain vibration signal analysis from experimental data and synthesis with frequency domain analysis techniques of finite element models to optimize the performance of an active magnetic bearing supporting an overhung disk rotor system.
\end{itemize}
This work will not deeply explore time domain signal processing, frequency domain analysis, or the finite element method, but is intended to provide an overview of these methods and how they pertain to rotating machinery. This work was developed with the intention to aid in the continued exploration of rotordynamics at Cal Poly.\par